\ifx \allfiles \undefined
\documentclass[12pt,a4paper,oneside]{report}

%% === CJK 套件 ===
\usepackage{CJKutf8,CJKnumb}                        % 中文套件
\usepackage[unicode]{hyperref,xcolor}
\hypersetup{
    colorlinks,
    linkcolor={blue!100!black},
    citecolor={blue!75!black},
    urlcolor={blue!50!black}
}
%% === AMS 標準套件 ===
\usepackage{amsmath,amsfonts,amssymb,amsthm}    % 數學符號
%% === 章節內容 ===
\usepackage{enumitem}                           % 修改 enumerate, item
\usepackage{titletoc,titlesec}                  % titletoc 目錄修改套件, titlesec 美化章節標題套件
\usepackage{imakeidx}                           % 索引
%% ===  ===
\usepackage[chapter]{algorithm}                 % 演算法套件
\usepackage[noend]{algpseudocode}               % pseudocode 套件
\usepackage{listings}                           % 程式碼
%% ===  ===
\usepackage{tikz,tkz-graph,tkz-berge}
%% ===  ===
\usepackage{xkeyval,xargs}

\makeindex[name=noun]        % 索引生成
\linespread{1.24}

%% === 設定頁面格式 ===
%\hoffset         = 10pt                      % 水平位移,預設為 0pt
\voffset         = -15pt                     % 垂直位移,預設為 0pt
\oddsidemargin   = 0pt                       % 預設為 31pt
%\topmargin       = 20pt                      % 預設為 20pt
%\headheight      = 12pt                      % header 的高度,預設為 12pt
%\headsep         = 25pt                      % header 和 body 的距離,預設為 25pt
\textheight      = 620pt                     % body 內文部分的高度,預設為 592pt
\textwidth       = 450pt                     % body 內文部分的寬度,預設為 390pt
%\marginparsep    = 10pt                      % margin note 和 body 的距離,預設為 10pt
%\marginparwidth  = 35pt                      % margin note 的寬度,預設為 35pt
%\footskip        = 30pt                      % footer 高度 + footer 和 body 的距離,預設為 30pt

%% === itemize,enumerate 設定 ===
%  使用 enumitem 套件
\setlist[itemize]{itemsep=0pt,parsep=0pt}
\setlist[enumerate]{itemsep=0pt,parsep=0pt}

\begin{document}
\begin{CJK}{UTF8}{bkai}
\newtheorem{mydef}{定義}[chapter]
\newtheorem*{mydef*}{定義}
\newtheorem{myrule}[mydef]{原理}
\newtheorem{mythm}[mydef]{定理}
\newtheorem{mylma}[mydef]{引理}
\newtheorem{mypropo}[mydef]{性質}
\newtheorem{mycorol}[mydef]{推論}
\newtheorem{myexample}[mydef]{範例}
\newtheorem*{mynote*}{註}
\numberwithin{equation}{section}
\renewenvironment{proof}{\textbf{證明}}{\qed}
\newenvironment{mysol}{\textbf{解答}}{\qed}

%% === 載入符號表 ===
%% === Basic Number Symbols ===
\providecommandx*{\PosInt}{\ensuremath{{\mathbb{Z}^{+}}}}
\providecommandx*{\NonNegInt}{\ensuremath{\mathbb{N}}}
\providecommandx*{\Int}{\ensuremath{\mathbb{Z}}}
\providecommandx*{\Ration}{\ensuremath{\mathbb{Q}}}
\providecommandx*{\Irration}{\ensuremath{\mathbb{Q}_{c}}}
\providecommandx*{\Real}{\ensuremath{\mathbb{R}}}
\providecommandx*{\Comp}{\ensuremath{\mathbb{C}}}
\providecommandx*{\Mat}[3][1=n,2=n]{\ensuremath{\mathbb{M}_{#1\times{#2}}{\left({#3}\right)}}}
\providecommandx*{\FuncSet}[3][1=\mathbb{F},2=\mathbb{R},3=\mathbb{R}]{\ensuremath{#1{\left({#2,#3}\right)}}}
%% ===  ===
\providecommandx*{\True}{\ensuremath{\top}}
\providecommandx*{\False}{\ensuremath{\bot}}
%% ===  ===
\providecommandx*{\Func}[3]{\ensuremath{#1:#2\rightarrow{#3}}}
\providecommandx*{\DefSet}[2]{\ensuremath{\left\{{#1|#2}\right\}}}
\providecommandx*{\NSet}[3][1=A,2=\times,3=n]{\ensuremath{{#1}_{1}#2{{#1}_{2}}#2\ldots{#2{#1}_{#3}}}}

%% ===  ===
\providecommandx*{\defaultSet}{\ensuremath{S}}
\providecommandx*{\defaultRelation}[2][1=,2=\mathcal{R}]{\ensuremath{{#2}_{#1}}}
\providecommandx*{\defaultFirstRelation}{\ensuremath{\defaultRelation[1]}}
\providecommandx*{\defaultSecondRelation}{\ensuremath{\defaultRelation[2]}}
\providecommandx*{\defaultTwoRelation}{\ensuremath{\defaultFirstRelation,\defaultSecondRelation}}
\providecommandx*{\defaultNRelation}[2][1=1,2=n]{\ensuremath{\defaultRelation[#1],\ldots{},\defaultRelation[#2]}}
%% === Algebra ===
\providecommandx*{\defaultAdd}[1][1=\defaultSet]{\ensuremath{\defaultRelation[#1][+]}}
\providecommandx*{\defaultTimes}[1][1=\defaultSet]{\ensuremath{\defaultRelation[#1][\cdot]}}
\providecommandx*{\defaultZero}[1][1=\defaultSet]{\ensuremath{{0}_{#1}}}
\providecommandx*{\defaultUnit}[1][1=\defaultSet]{\ensuremath{{1}_{#1}}}
\providecommandx*{\Algebra}[2][1=\defaultSet,2=\defaultRelation]{\ensuremath{\left({#1,#2}\right)}}
\providecommandx*{\AddAlgebra}[1][1=\defaultSet]{\ensuremath{\Algebra[#1][{\defaultAdd[#1]}]}}
\providecommandx*{\TimesAlgebra}[1][1=\defaultSet]{\ensuremath{\Algebra[#1][{\defaultTimes[#1]}]}}
\providecommandx*{\GeneralAlgebra}[2][1=\defaultSet,2=\defaultNRelation]{\ensuremath{\Algebra[#1][#2]}}
\providecommandx*{\Op}[3][1=\defaultRelation]{\ensuremath{#2#1#3}}
%% === Group ===
\providecommandx*{\GroupSet}[1][1=G]{\ensuremath{#1}}
\providecommandx*{\GroupRelation}[2][1=\GroupSet,2=\cdot]{\ensuremath{#2_{#1}}}
\providecommandx*{\Group}[2][1=\GroupSet,2=\GroupRelation]{\ensuremath{\Algebra[#1][{#2[#1]}]}}
\providecommandx*{\GOp}[3][1=\GroupRelation]{\ensuremath{\Op[#1]{#2}{#3}}}
\fi

\chapter{基礎知識}

\section{數系}

\begin{mydef}[常見數系定義]
\label{def:intro:number_set_notation}
對於一般數字,我們定義以下集合:
\begin{enumerate}
\item \textbf{非負整數}表示為 $\NonNegInt={\{{0,1,2,\cdots}\}}$
\item \textbf{整數}表示為 $\Int={\{{\cdots,-1,0,1,\cdots}\}}$
\item \textbf{正整數}表示為 $\PosInt={\{1,2,\cdots\}}$
\item \textbf{負整數}表示為 $\NegInt={\{-1,-2,\cdots\}}$
\item \textbf{有理數}表示為 $\Ration$
\item \textbf{無理數}表示為 $\Irration$
\item \textbf{實數}表示為 $\Real$
\item \textbf{正數}表示為 $\PosReal$
\item \textbf{負數}表示為 $\NegReal$
\item \textbf{複數}表示為 $\Comp$
\end{enumerate}
\end{mydef}

\begin{mydef}[雙複數]
定義\textbf{雙複數 (Bicomplex number)} $a+bi+cj+dk$,其中 $a,b,c,d\in{\Real}$,並滿足以下運算法則:
\begin{enumerate}
\item $ij=ji=k$
\item $i^2=k^2=-1$
\item $j^2=1$
\end{enumerate}
\end{mydef}

\begin{mydef}[四元數]
定義\textbf{四元數 (Quaternion)} $a+bi+cj+dk$,其中 $a,b,c,d\in{\Real}$,並滿足運算法則 $i^2=j^2=k^2=ijk=-1$。
\end{mydef}

\section{函數}

\begin{mydef}[函數]
\label{def:intro:function_notation}
有兩個集合 $A$、$B$,若 $A$ 和 $B$ 之間存在對應關係 $f$,使得所有 $a\in{A}$ 皆可對應到\textbf{唯一}的 $b\in{B}$,則 $f$ 稱為\textbf{函數 (Function)},記為 $\Func{f}{A}{B}$,此時 $a$ 對應到 $b$ 記為 $f(a)=b$。
\end{mydef}
\begin{mynote*}
對所有 $a\in{A}$ 不會對應到兩個以上的 $b$,或是沒有 $b$ 值。
\end{mynote*}

\begin{mydef}[定義域、對應域與值域]
\label{def:intro:domain_codomain_and_range}
有兩個集合 $A$、$B$,若有一函數 $\Func{f}{A}{B}$,則
\begin{enumerate}
\item $A$ 稱為\textbf{定義域 (Domain)}
\item $B$ 稱為\textbf{對應域 (Codomain)}
\item 對於所有 $a\in{A}$,$f(a)$ 形成的集合稱為\textbf{值域 (Range)},記為 $f(A)$ 或是 $R(f)$。
\end{enumerate}
\end{mydef}

\begin{mydef}[函數的合成運算]
\label{def:intro:function_composition}
有三個集合 $A$、$B$ 和 $C$,若有兩個函數 $f$、$g$
\begin{align*}
\Func{f}{A}{B}\\
\Func{g}{B}{C}
\end{align*}
定義函數的合成運算 $h=g\circ{f}$ (簡寫為 $gf$),其中 $\Func{h}{A}{C}$,使得對所有 $a\in{A}$
\begin{align*}
h(a)={{(gf)}{(a)}}={{({g\circ{f}})}{(a)}}={g({f(a)})}
\end{align*}
\end{mydef}

\begin{mythm}[函數合成結合律]
\label{thm:intro:function_composition_associativity}
有四個集合 $A_1$、$A_2$、$A_3$ 和 $A_4$,若有三個函數 $f$、$g$、$h$
\begin{align*}
\Func{f}{A_1}{A_2}\\
\Func{g}{A_2}{A_3}\\
\Func{h}{A_3}{A_4}
\end{align*}
則函數合成符合 $(hg)f=h(gf)$。
\end{mythm}
\begin{proof}
對所有 $a_1\in{A_1}$,則
\begin{align*}
{((hg)f)}{(a_1)} &= {(hg)({f(a_1)})}   &\text{根據定義 \ref{def:intro:function_composition}}\\
                 &= {h({g({f(a_1)})})} &\text{根據定義 \ref{def:intro:function_composition}}\\
                 &= h({{(gf)}{(a_1)}}) &\text{根據定義 \ref{def:intro:function_composition}}\\
                 &= {(h(gf))}{(a_1)}   &\text{根據定義 \ref{def:intro:function_composition}}
\end{align*}
\end{proof}

\begin{mydef}[單位函數]
\label{def:intro:identity_function}
集合 $A$ 上有一函數 $\Func{I_A}{A}{A}$,若對於所有 $a\in{A}$ 使得 $I_A{(a)}=a$,則 $I_A$ 稱為在 $A$ 上的\textbf{單位函數 (Identity function)}。
\end{mydef}

\begin{mydef}[反函數]
\label{def:intro:inverse_function}
兩集合 $A$、$B$ 上有一函數 $\Func{f}{A}{B}$,若存在 $\Func{g}{B}{A}$ 使得
\begin{align*}
gf=I_A\\
fg=I_B
\end{align*}
則 $g$ 稱為 $f$ 的反函數,此時 $g$ 可記為 $f^{-1}$。
\end{mydef}

\begin{mydef}[可逆函數]
\label{def:intro:invertible_function}
兩集合 $A$、$B$ 上有一函數 $\Func{f}{A}{B}$,若 $f$ 存在反函數,則 $f$ 稱為\textbf{可逆函數 (Invertible function)}。
\end{mydef}

\begin{mydef}[單射函數]
\label{def:intro:injective_function}
兩集合 $A$、$B$ 上有一函數 $\Func{f}{A}{B}$,若對於所有 $a_1,a_2\in{A}$,$f(a_1)=f(a_2)$ 可以得到 $a_1=a_2$,我們稱函數 $f$ 為\textbf{一對一函數 (One-to-one function)} 或是\textbf{單射函數 (Injective function)}。
\end{mydef}

\begin{mydef}[滿射函數]
\label{def:intro:surjective_function}
兩集合 $A$、$B$ 上有一函數 $\Func{f}{A}{B}$,若對於所有 $b\in{B}$ 皆可找到 $a\in{A}$ 使得 $f(a)=b$,我們稱函數 $f$ 為\textbf{映成函數 (Onto function)} 或是\textbf{滿射函數 (Surjective function)}。
\end{mydef}

\begin{mydef}[雙射函數]
\label{def:intro:bijective_function}
兩集合 $A$、$B$ 上有一函數 $\Func{f}{A}{B}$,若 $f$ 是單射函數且為滿射函數,則 $f$ 稱為\textbf{雙射函數 (Bijective function)}。
\end{mydef}

\begin{mythm}
\label{thm:intro:bijective_iff_invertible}
\label{exe:intro:bijective_iff_invertible}
兩集合 $A$、$B$ 上有一函數 $\Func{f}{A}{B}$,則 $f$ 是可逆函數若且唯若 $f$ 是雙射函數。
\end{mythm}
\begin{proof}
做為習題。
\end{proof}

\begin{mythm}[反函數唯一性]
\label{thm:intro:inverse_function_uniqueness}
\label{exe:intro:inverse_function_uniqueness}
兩集合 $A$、$B$ 上有一函數 $\Func{f}{A}{B}$,若 $f$ 是可逆函數,則 $f$ 的反函數是唯一的。
\end{mythm}
\begin{proof}
做為習題。
\end{proof}

\begin{mythm}
\label{thm:intro:inverse_function_invertible}
\label{exe:intro:inverse_function_invertible}
兩集合 $A$、$B$ 上有一函數 $\Func{f}{A}{B}$,若 $f$ 是可逆函數,則反函數 $\Func{f^{-1}}{B}{A}$ 也是可逆函數。
\end{mythm}
\begin{proof}
做為習題。
\end{proof}

\section*{習題}

\begin{enumerate}
\item 證明定理 \ref{exe:intro:bijective_iff_invertible}。
\item 證明定理 \ref{exe:intro:inverse_function_uniqueness}。
\item 證明定理 \ref{exe:intro:inverse_function_invertible}。
\end{enumerate}

\ifx \allfiles \undefined
\printindex[noun]

\clearpage
\end{CJK}
\end{document}
\fi