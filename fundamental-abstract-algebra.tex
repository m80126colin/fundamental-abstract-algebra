\documentclass[12pt,a4paper,oneside]{book}
\def \allfiles {}

%% === CJK 套件 ===
\usepackage{CJKutf8,CJKnumb}                     % 中文套件
\usepackage[unicode]{hyperref,xcolor}
\hypersetup{
    colorlinks,
    linkcolor={blue!100!black},
    citecolor={blue!75!black},
    urlcolor={blue!50!black}
}
%% === AMS 標準套件 ===
\usepackage{amsmath,amsfonts,amssymb,amsthm} % 數學符號
%% === 章節內容 ===
\usepackage{enumitem}                        % 修改 enumerate, item
\usepackage{titletoc,titlesec}               % titletoc 目錄修改套件, titlesec 美化章節標題套件
\usepackage{imakeidx}                        % 索引
%% ===  ===
\usepackage[chapter]{algorithm}              % 演算法套件
\usepackage[noend]{algpseudocode}            % pseudocode 套件
\usepackage{listings}                        % 程式碼
%% === TikZ 套件 ===
\usepackage{tikz,tkz-graph,tkz-berge}        % 繪圖
\usepackage{xkeyval,xargs}

\linespread{1.24}

%% === itemize,enumerate 設定 ===
%  使用 enumitem 套件
\setlist[itemize]{itemsep=0pt,parsep=0pt}
\setlist[enumerate]{itemsep=0pt,parsep=0pt}


%% === 設定頁面格式 ===
%\hoffset         = 10pt                      % 水平位移,預設為 0pt
\voffset         = -15pt                     % 垂直位移,預設為 0pt
\oddsidemargin   = 0pt                       % 預設為 31pt
%\topmargin       = 20pt                      % 預設為 20pt
%\headheight      = 12pt                      % header 的高度,預設為 12pt
%\headsep         = 25pt                      % header 和 body 的距離,預設為 25pt
\textheight      = 620pt                     % body 內文部分的高度,預設為 592pt
\textwidth       = 450pt                     % body 內文部分的寬度,預設為 390pt
%\marginparsep    = 10pt                      % margin note 和 body 的距離,預設為 10pt
%\marginparwidth  = 35pt                      % margin note 的寬度,預設為 35pt
%\footskip        = 30pt                      % footer 高度 + footer 和 body 的距離,預設為 30pt

\makeindex[name=noun]        % 索引生成

\begin{document}
\begin{CJK}{UTF8}{bkai}

%% === 常用的指令,替換成中文 ===
\renewcommand{\figurename}{圖}
\renewcommand{\tablename}{表}
\renewcommand{\contentsname}{目~錄}
\renewcommand{\listfigurename}{插~圖~目~錄}
\renewcommand{\listtablename}{表~格~目~錄}
\renewcommand{\appendixname}{附~錄}
%\renewcommand{\refname}{參~考~資~料}    % article
\renewcommand{\bibname}{參~考~文~獻}     % book
\renewcommand{\indexname}{索~引}
\renewcommand{\today}{\number\year~年~\number\month~月~\number\day~日}
%\newcommand{\zhtoday}{\CJKdigits{\the\year}年\CJKnumber{\the\month}月\CJKnumber{\the\day}日}

%% === 
\floatname{algorithm}{演算法}

%% === counter 設定 ===
\setcounter{secnumdepth}{3}                                 % 設定計數到 subsubsection
\renewcommand{\thepart}{第\CJKnumber{\arabic{part}}部分}
\renewcommand{\thechapter}{\arabic{chapter}}
\renewcommand{\thesection}{\arabic{section}}                % 改 section 為 1, 2, 3 非 1.1, 1.2, 1.3
\renewcommand{\thesubsection}{\arabic{subsection}}          % subsection 也改一改
\renewcommand{\thesubsubsection}{\arabic{subsubsection}}    % subsubsection 也改一改

%% === 設定章節標題 (配合 titlesec) ===
\titleformat{\part}[display]
	{\center\huge\bfseries}
	{\thepart}
	{1em}
	{\Huge}
%
\titleformat{\chapter}[display]
	{\bf\Large}
	{第~\CJKnumber{\thechapter}~章}
	{1ex}
	{\Huge}
	[\vspace{2ex}]
%
\titleformat{\section}{\Large\bfseries}{第\CJKnumber{\thesection}節}{1em}{}
%
\titleformat{\subsection}{\large\bfseries}{\CJKnumber{\thesubsection}、}{0.5em}{}
%
\titleformat{\subsubsection}{\bfseries}{(\CJKnumber{\thesubsubsection})}{0.5em}{}
% \titlespacing{\subsubsection}{0pt}{5pt}{-10pt}

%% === 設定目錄標題 (配合 titletoc) ===
\titlecontents{part}[0em]
{\center\Large}
{}
{}
{}
%
\titlecontents{chapter}[0em]
{}
{\large\bf{第\CJKnumber{\thecontentslabel}章~~}}
{}{~~\titlerule*{.}\bf\contentspage}
%
\titlecontents{section}[4em]
{}
{第\CJKnumber{\thecontentslabel}節\quad}
{}{~~\titlerule*{.} \contentspage}
%
\titlecontents{subsection}[8em]
{}
{\CJKnumber{\thecontentslabel}、}
{}{~~\titlerule*{.} \contentspage}

%% === 定義 ===
\newtheorem{mydef}{定義}[chapter]
\newtheorem*{mydef*}{定義}
\newtheorem{myrule}[mydef]{原理}
\newtheorem{mythm}[mydef]{定理}
\newtheorem{mypropo}[mydef]{性質}
\newtheorem{mycorol}[mydef]{推論}
\newtheorem{myexample}[mydef]{範例}
\newtheorem*{mynote*}{註}
\numberwithin{equation}{section}
\renewenvironment{proof}{\textbf{證明}}{\qed}
\newenvironment{mysol}{\textbf{解答}}{\qed}

%% === 載入符號表 ===
%% === Basic Number Symbols ===
\providecommandx*{\PosInt}{\ensuremath{{\mathbb{Z}^{+}}}}
\providecommandx*{\NonNegInt}{\ensuremath{\mathbb{N}}}
\providecommandx*{\Int}{\ensuremath{\mathbb{Z}}}
\providecommandx*{\Ration}{\ensuremath{\mathbb{Q}}}
\providecommandx*{\Irration}{\ensuremath{\mathbb{Q}_{c}}}
\providecommandx*{\Real}{\ensuremath{\mathbb{R}}}
\providecommandx*{\Comp}{\ensuremath{\mathbb{C}}}
\providecommandx*{\Mat}[3][1=n,2=n]{\ensuremath{\mathbb{M}_{#1\times{#2}}{\left({#3}\right)}}}
\providecommandx*{\FuncSet}[3][1=\mathbb{F},2=\mathbb{R},3=\mathbb{R}]{\ensuremath{#1{\left({#2,#3}\right)}}}
%% ===  ===
\providecommandx*{\True}{\ensuremath{\top}}
\providecommandx*{\False}{\ensuremath{\bot}}
%% ===  ===
\providecommandx*{\Func}[3]{\ensuremath{#1:#2\rightarrow{#3}}}
\providecommandx*{\DefSet}[2]{\ensuremath{\left\{{#1|#2}\right\}}}
\providecommandx*{\NSet}[3][1=A,2=\times,3=n]{\ensuremath{{#1}_{1}#2{{#1}_{2}}#2\ldots{#2{#1}_{#3}}}}

%% ===  ===
\providecommandx*{\defaultSet}{\ensuremath{S}}
\providecommandx*{\defaultRelation}[2][1=,2=\mathcal{R}]{\ensuremath{{#2}_{#1}}}
\providecommandx*{\defaultFirstRelation}{\ensuremath{\defaultRelation[1]}}
\providecommandx*{\defaultSecondRelation}{\ensuremath{\defaultRelation[2]}}
\providecommandx*{\defaultTwoRelation}{\ensuremath{\defaultFirstRelation,\defaultSecondRelation}}
\providecommandx*{\defaultNRelation}[2][1=1,2=n]{\ensuremath{\defaultRelation[#1],\ldots{},\defaultRelation[#2]}}
%% === Algebra ===
\providecommandx*{\defaultAdd}[1][1=\defaultSet]{\ensuremath{\defaultRelation[#1][+]}}
\providecommandx*{\defaultTimes}[1][1=\defaultSet]{\ensuremath{\defaultRelation[#1][\cdot]}}
\providecommandx*{\defaultZero}[1][1=\defaultSet]{\ensuremath{{0}_{#1}}}
\providecommandx*{\defaultUnit}[1][1=\defaultSet]{\ensuremath{{1}_{#1}}}
\providecommandx*{\Algebra}[2][1=\defaultSet,2=\defaultRelation]{\ensuremath{\left({#1,#2}\right)}}
\providecommandx*{\AddAlgebra}[1][1=\defaultSet]{\ensuremath{\Algebra[#1][{\defaultAdd[#1]}]}}
\providecommandx*{\TimesAlgebra}[1][1=\defaultSet]{\ensuremath{\Algebra[#1][{\defaultTimes[#1]}]}}
\providecommandx*{\GeneralAlgebra}[2][1=\defaultSet,2=\defaultNRelation]{\ensuremath{\Algebra[#1][#2]}}
\providecommandx*{\Op}[3][1=\defaultRelation]{\ensuremath{#2#1#3}}
%% === Group ===
\providecommandx*{\GroupSet}[1][1=G]{\ensuremath{#1}}
\providecommandx*{\GroupRelation}[2][1=\GroupSet,2=\cdot]{\ensuremath{#2_{#1}}}
\providecommandx*{\Group}[2][1=\GroupSet,2=\GroupRelation]{\ensuremath{\Algebra[#1][{#2[#1]}]}}
\providecommandx*{\GOp}[3][1=\GroupRelation]{\ensuremath{\Op[#1]{#2}{#3}}}

\title{Fundamental Abstract Algebra\\基礎抽象代數}
\author{許胖}
\maketitle
\tableofcontents

\ifx \allfiles \undefined
\documentclass[12pt,a4paper,oneside]{report}

%% === CJK 套件 ===
\usepackage{CJKutf8,CJKnumb}                        % 中文套件
\usepackage[unicode]{hyperref,xcolor}
\hypersetup{
    colorlinks,
    linkcolor={blue!100!black},
    citecolor={blue!75!black},
    urlcolor={blue!50!black}
}
%% === AMS 標準套件 ===
\usepackage{amsmath,amsfonts,amssymb,amsthm}    % 數學符號
%% === 章節內容 ===
\usepackage{enumitem}                           % 修改 enumerate, item
\usepackage{titletoc,titlesec}                  % titletoc 目錄修改套件, titlesec 美化章節標題套件
\usepackage{imakeidx}                           % 索引
%% ===  ===
\usepackage[chapter]{algorithm}                 % 演算法套件
\usepackage[noend]{algpseudocode}               % pseudocode 套件
\usepackage{listings}                           % 程式碼
%% ===  ===
\usepackage{tikz,tkz-graph,tkz-berge}
%% ===  ===
\usepackage{xkeyval,xargs}

\makeindex[name=noun]        % 索引生成
\linespread{1.24}

%% === 設定頁面格式 ===
%\hoffset         = 10pt                      % 水平位移,預設為 0pt
\voffset         = -15pt                     % 垂直位移,預設為 0pt
\oddsidemargin   = 0pt                       % 預設為 31pt
%\topmargin       = 20pt                      % 預設為 20pt
%\headheight      = 12pt                      % header 的高度,預設為 12pt
%\headsep         = 25pt                      % header 和 body 的距離,預設為 25pt
\textheight      = 620pt                     % body 內文部分的高度,預設為 592pt
\textwidth       = 450pt                     % body 內文部分的寬度,預設為 390pt
%\marginparsep    = 10pt                      % margin note 和 body 的距離,預設為 10pt
%\marginparwidth  = 35pt                      % margin note 的寬度,預設為 35pt
%\footskip        = 30pt                      % footer 高度 + footer 和 body 的距離,預設為 30pt

%% === itemize,enumerate 設定 ===
%  使用 enumitem 套件
\setlist[itemize]{itemsep=0pt,parsep=0pt}
\setlist[enumerate]{itemsep=0pt,parsep=0pt}

\begin{document}
\begin{CJK}{UTF8}{bkai}
\newtheorem{mydef}{定義}[chapter]
\newtheorem*{mydef*}{定義}
\newtheorem{myrule}[mydef]{原理}
\newtheorem{mythm}[mydef]{定理}
\newtheorem{mypropo}[mydef]{性質}
\newtheorem{mycorol}[mydef]{推論}
\newtheorem{myexample}[mydef]{範例}
\newtheorem*{mynote*}{註}
\numberwithin{equation}{section}
\renewenvironment{proof}{\textbf{證明}}{\qed}
\newenvironment{mysol}{\textbf{解答}}{\qed}

%% === 載入符號表 ===
%% === Basic Number Symbols ===
\providecommandx*{\PosInt}{\ensuremath{{\mathbb{Z}^{+}}}}
\providecommandx*{\NonNegInt}{\ensuremath{\mathbb{N}}}
\providecommandx*{\Int}{\ensuremath{\mathbb{Z}}}
\providecommandx*{\Ration}{\ensuremath{\mathbb{Q}}}
\providecommandx*{\Irration}{\ensuremath{\mathbb{Q}_{c}}}
\providecommandx*{\Real}{\ensuremath{\mathbb{R}}}
\providecommandx*{\Comp}{\ensuremath{\mathbb{C}}}
\providecommandx*{\Mat}[3][1=n,2=n]{\ensuremath{\mathbb{M}_{#1\times{#2}}{\left({#3}\right)}}}
\providecommandx*{\FuncSet}[3][1=\mathbb{F},2=\mathbb{R},3=\mathbb{R}]{\ensuremath{#1{\left({#2,#3}\right)}}}
%% ===  ===
\providecommandx*{\True}{\ensuremath{\top}}
\providecommandx*{\False}{\ensuremath{\bot}}
%% ===  ===
\providecommandx*{\Func}[3]{\ensuremath{#1:#2\rightarrow{#3}}}
\providecommandx*{\DefSet}[2]{\ensuremath{\left\{{#1|#2}\right\}}}
\providecommandx*{\NSet}[3][1=A,2=\times,3=n]{\ensuremath{{#1}_{1}#2{{#1}_{2}}#2\ldots{#2{#1}_{#3}}}}

%% ===  ===
\providecommandx*{\defaultSet}{\ensuremath{S}}
\providecommandx*{\defaultRelation}[2][1=,2=\mathcal{R}]{\ensuremath{{#2}_{#1}}}
\providecommandx*{\defaultFirstRelation}{\ensuremath{\defaultRelation[1]}}
\providecommandx*{\defaultSecondRelation}{\ensuremath{\defaultRelation[2]}}
\providecommandx*{\defaultTwoRelation}{\ensuremath{\defaultFirstRelation,\defaultSecondRelation}}
\providecommandx*{\defaultNRelation}[2][1=1,2=n]{\ensuremath{\defaultRelation[#1],\ldots{},\defaultRelation[#2]}}
%% === Algebra ===
\providecommandx*{\defaultAdd}[1][1=\defaultSet]{\ensuremath{\defaultRelation[#1][+]}}
\providecommandx*{\defaultTimes}[1][1=\defaultSet]{\ensuremath{\defaultRelation[#1][\cdot]}}
\providecommandx*{\defaultZero}[1][1=\defaultSet]{\ensuremath{{0}_{#1}}}
\providecommandx*{\defaultUnit}[1][1=\defaultSet]{\ensuremath{{1}_{#1}}}
\providecommandx*{\Algebra}[2][1=\defaultSet,2=\defaultRelation]{\ensuremath{\left({#1,#2}\right)}}
\providecommandx*{\AddAlgebra}[1][1=\defaultSet]{\ensuremath{\Algebra[#1][{\defaultAdd[#1]}]}}
\providecommandx*{\TimesAlgebra}[1][1=\defaultSet]{\ensuremath{\Algebra[#1][{\defaultTimes[#1]}]}}
\providecommandx*{\GeneralAlgebra}[2][1=\defaultSet,2=\defaultNRelation]{\ensuremath{\Algebra[#1][#2]}}
\providecommandx*{\Op}[3][1=\defaultRelation]{\ensuremath{#2#1#3}}
%% === Group ===
\providecommandx*{\GroupSet}[1][1=G]{\ensuremath{#1}}
\providecommandx*{\GroupRelation}[2][1=\GroupSet,2=\cdot]{\ensuremath{#2_{#1}}}
\providecommandx*{\Group}[2][1=\GroupSet,2=\GroupRelation]{\ensuremath{\Algebra[#1][{#2[#1]}]}}
\providecommandx*{\GOp}[3][1=\GroupRelation]{\ensuremath{\Op[#1]{#2}{#3}}}
\fi

\chapter{基礎知識}

\section{數系}

\begin{mydef}[常見數系定義]
\label{def:intro:number_set_notation}
對於一般數字,我們定義以下集合:
\begin{enumerate}
\item \textbf{非負整數}表示為 $\NonNegInt={\{{0,1,2,\cdots}\}}$
\item \textbf{整數}表示為 $\Int={\{{\cdots,-1,0,1,\cdots}\}}$
\item \textbf{正整數}表示為 $\PosInt={\{1,2,\cdots\}}$
\item \textbf{負整數}表示為 $\NegInt={\{-1,-2,\cdots\}}$
\item \textbf{有理數}表示為 $\Ration$
\item \textbf{無理數}表示為 $\Irration$
\item \textbf{實數}表示為 $\Real$
\item \textbf{正數}表示為 $\PosReal$
\item \textbf{負數}表示為 $\NegReal$
\item \textbf{複數}表示為 $\Comp$
\end{enumerate}
\end{mydef}

\begin{mydef}[雙複數]
定義\textbf{雙複數 (Bicomplex number)} $a+bi+cj+dk$,其中 $a,b,c,d\in{\Real}$,並滿足以下運算法則:
\begin{enumerate}
\item $ij=ji=k$
\item $i^2=k^2=-1$
\item $j^2=1$
\end{enumerate}
\end{mydef}

\begin{mydef}[四元數]
定義\textbf{四元數 (Quaternion)} $a+bi+cj+dk$,其中 $a,b,c,d\in{\Real}$,並滿足運算法則 $i^2=j^2=k^2=ijk=-1$。
\end{mydef}

\section{函數}

\begin{mydef}[函數]
\label{def:intro:function_notation}
有兩個集合 $A$、$B$,若 $A$ 和 $B$ 之間存在對應關係 $f$,使得所有 $a\in{A}$ 皆可對應到\textbf{唯一}的 $b\in{B}$,則 $f$ 稱為\textbf{函數 (Function)},記為 $\Func{f}{A}{B}$,此時 $a$ 對應到 $b$ 記為 $f(a)=b$。
\end{mydef}
\begin{mynote*}
對所有 $a\in{A}$ 不會對應到兩個以上的 $b$,或是沒有 $b$ 值。
\end{mynote*}

\begin{mydef}[定義域、對應域與值域]
\label{def:intro:domain_codomain_and_range}
有兩個集合 $A$、$B$,若有一函數 $\Func{f}{A}{B}$,則
\begin{enumerate}
\item $A$ 稱為\textbf{定義域 (Domain)}
\item $B$ 稱為\textbf{對應域 (Codomain)}
\item 對於所有 $a\in{A}$,$f(a)$ 形成的集合稱為\textbf{值域 (Range)},記為 $f(A)$ 或是 $R(f)$。
\end{enumerate}
\end{mydef}

\begin{mydef}[函數的合成運算]
\label{def:intro:function_composition}
有三個集合 $A$、$B$ 和 $C$,若有兩個函數 $f$、$g$
\begin{align*}
\Func{f}{A}{B}\\
\Func{g}{B}{C}
\end{align*}
定義函數的合成運算 $h=g\circ{f}$ (簡寫為 $gf$),其中 $\Func{h}{A}{C}$,使得對所有 $a\in{A}$
\begin{align*}
h(a)={{(gf)}{(a)}}={{({g\circ{f}})}{(a)}}={g({f(a)})}
\end{align*}
\end{mydef}

\begin{mythm}[函數合成結合律]
\label{thm:intro:function_composition_associativity}
有四個集合 $A_1$、$A_2$、$A_3$ 和 $A_4$,若有三個函數 $f$、$g$、$h$
\begin{align*}
\Func{f}{A_1}{A_2}\\
\Func{g}{A_2}{A_3}\\
\Func{h}{A_3}{A_4}
\end{align*}
則函數合成符合 $(hg)f=h(gf)$。
\end{mythm}
\begin{proof}
對所有 $a_1\in{A_1}$,則
\begin{align*}
{((hg)f)}{(a_1)} &= {(hg)({f(a_1)})}   &\text{根據定義 \ref{def:intro:function_composition}}\\
                 &= {h({g({f(a_1)})})} &\text{根據定義 \ref{def:intro:function_composition}}\\
                 &= h({{(gf)}{(a_1)}}) &\text{根據定義 \ref{def:intro:function_composition}}\\
                 &= {(h(gf))}{(a_1)}   &\text{根據定義 \ref{def:intro:function_composition}}
\end{align*}
\end{proof}

\begin{mydef}[單位函數]
\label{def:intro:identity_function}
集合 $A$ 上有一函數 $\Func{I_A}{A}{A}$,若對於所有 $a\in{A}$ 使得 $I_A{(a)}=a$,則 $I_A$ 稱為在 $A$ 上的\textbf{單位函數 (Identity function)}。
\end{mydef}

\begin{mydef}[反函數]
\label{def:intro:inverse_function}
兩集合 $A$、$B$ 上有一函數 $\Func{f}{A}{B}$,若存在 $\Func{g}{B}{A}$ 使得
\begin{align*}
gf=I_A\\
fg=I_B
\end{align*}
則 $g$ 稱為 $f$ 的反函數,此時 $g$ 可記為 $f^{-1}$。
\end{mydef}

\begin{mydef}[可逆函數]
\label{def:intro:invertible_function}
兩集合 $A$、$B$ 上有一函數 $\Func{f}{A}{B}$,若 $f$ 存在反函數,則 $f$ 稱為\textbf{可逆函數 (Invertible function)}。
\end{mydef}

\begin{mydef}[單射函數]
\label{def:intro:injective_function}
兩集合 $A$、$B$ 上有一函數 $\Func{f}{A}{B}$,若對於所有 $a_1,a_2\in{A}$,$f(a_1)=f(a_2)$ 可以得到 $a_1=a_2$,我們稱函數 $f$ 為\textbf{一對一函數 (One-to-one function)} 或是\textbf{單射函數 (Injective function)}。
\end{mydef}

\begin{mydef}[滿射函數]
\label{def:intro:surjective_function}
兩集合 $A$、$B$ 上有一函數 $\Func{f}{A}{B}$,若對於所有 $b\in{B}$ 皆可找到 $a\in{A}$ 使得 $f(a)=b$,我們稱函數 $f$ 為\textbf{映成函數 (Onto function)} 或是\textbf{滿射函數 (Surjective function)}。
\end{mydef}

\begin{mydef}[雙射函數]
\label{def:intro:bijective_function}
兩集合 $A$、$B$ 上有一函數 $\Func{f}{A}{B}$,若 $f$ 是單射函數且為滿射函數,則 $f$ 稱為\textbf{雙射函數 (Bijective function)}。
\end{mydef}

\begin{mythm}
\label{thm:intro:bijective_iff_invertible}
\label{exe:intro:bijective_iff_invertible}
兩集合 $A$、$B$ 上有一函數 $\Func{f}{A}{B}$,則 $f$ 是可逆函數若且唯若 $f$ 是雙射函數。
\end{mythm}
\begin{proof}
做為習題。
\end{proof}

\section*{習題}

\begin{enumerate}
\item 證明定理 \ref{exe:intro:bijective_iff_invertible}。
\end{enumerate}

\ifx \allfiles \undefined
\printindex[noun]

\clearpage
\end{CJK}
\end{document}
\fi
\ifx \allfiles \undefined
\documentclass[12pt,a4paper,oneside]{report}

%% === CJK 套件 ===
\usepackage{CJKutf8,CJKnumb}                        % 中文套件
\usepackage[unicode]{hyperref,xcolor}
\hypersetup{
    colorlinks,
    linkcolor={blue!100!black},
    citecolor={blue!75!black},
    urlcolor={blue!50!black}
}
%% === AMS 標準套件 ===
\usepackage{amsmath,amsfonts,amssymb,amsthm}    % 數學符號
%% === 章節內容 ===
\usepackage{enumitem}                           % 修改 enumerate, item
\usepackage{titletoc,titlesec}                  % titletoc 目錄修改套件, titlesec 美化章節標題套件
\usepackage{imakeidx}                           % 索引
%% ===  ===
\usepackage[chapter]{algorithm}                 % 演算法套件
\usepackage[noend]{algpseudocode}               % pseudocode 套件
\usepackage{listings}                           % 程式碼
%% ===  ===
\usepackage{tikz,tkz-graph,tkz-berge}
%% ===  ===
\usepackage{xkeyval,xargs}

\makeindex[name=noun]        % 索引生成
\linespread{1.24}

%% === 設定頁面格式 ===
%\hoffset         = 10pt                      % 水平位移,預設為 0pt
\voffset         = -15pt                     % 垂直位移,預設為 0pt
\oddsidemargin   = 0pt                       % 預設為 31pt
%\topmargin       = 20pt                      % 預設為 20pt
%\headheight      = 12pt                      % header 的高度,預設為 12pt
%\headsep         = 25pt                      % header 和 body 的距離,預設為 25pt
\textheight      = 620pt                     % body 內文部分的高度,預設為 592pt
\textwidth       = 450pt                     % body 內文部分的寬度,預設為 390pt
%\marginparsep    = 10pt                      % margin note 和 body 的距離,預設為 10pt
%\marginparwidth  = 35pt                      % margin note 的寬度,預設為 35pt
%\footskip        = 30pt                      % footer 高度 + footer 和 body 的距離,預設為 30pt

%% === itemize,enumerate 設定 ===
%  使用 enumitem 套件
\setlist[itemize]{itemsep=0pt,parsep=0pt}
\setlist[enumerate]{itemsep=0pt,parsep=0pt}

\begin{document}
\begin{CJK}{UTF8}{bkai}
\newtheorem{mydef}{定義}[chapter]
\newtheorem*{mydef*}{定義}
\newtheorem{myrule}[mydef]{原理}
\newtheorem{mythm}[mydef]{定理}
\newtheorem{mylma}[mydef]{引理}
\newtheorem{mypropo}[mydef]{性質}
\newtheorem{mycorol}[mydef]{推論}
\newtheorem{myexample}[mydef]{範例}
\newtheorem*{mynote*}{註}
\numberwithin{equation}{section}
\renewenvironment{proof}{\textbf{證明}}{\qed}
\newenvironment{mysol}{\textbf{解答}}{\qed}

%% === 載入符號表 ===
%% === Basic Number Symbols ===
\providecommandx*{\PosInt}{\ensuremath{{\mathbb{Z}^{+}}}}
\providecommandx*{\NonNegInt}{\ensuremath{\mathbb{N}}}
\providecommandx*{\Int}{\ensuremath{\mathbb{Z}}}
\providecommandx*{\Ration}{\ensuremath{\mathbb{Q}}}
\providecommandx*{\Irration}{\ensuremath{\mathbb{Q}_{c}}}
\providecommandx*{\Real}{\ensuremath{\mathbb{R}}}
\providecommandx*{\Comp}{\ensuremath{\mathbb{C}}}
\providecommandx*{\Mat}[3][1=n,2=n]{\ensuremath{\mathbb{M}_{#1\times{#2}}{\left({#3}\right)}}}
\providecommandx*{\FuncSet}[3][1=\mathbb{F},2=\mathbb{R},3=\mathbb{R}]{\ensuremath{#1{\left({#2,#3}\right)}}}
%% ===  ===
\providecommandx*{\True}{\ensuremath{\top}}
\providecommandx*{\False}{\ensuremath{\bot}}
%% ===  ===
\providecommandx*{\Func}[3]{\ensuremath{#1:#2\rightarrow{#3}}}
\providecommandx*{\DefSet}[2]{\ensuremath{\left\{{#1|#2}\right\}}}
\providecommandx*{\NSet}[3][1=A,2=\times,3=n]{\ensuremath{{#1}_{1}#2{{#1}_{2}}#2\ldots{#2{#1}_{#3}}}}

%% ===  ===
\providecommandx*{\defaultSet}{\ensuremath{S}}
\providecommandx*{\defaultRelation}[2][1=,2=\mathcal{R}]{\ensuremath{{#2}_{#1}}}
\providecommandx*{\defaultFirstRelation}{\ensuremath{\defaultRelation[1]}}
\providecommandx*{\defaultSecondRelation}{\ensuremath{\defaultRelation[2]}}
\providecommandx*{\defaultTwoRelation}{\ensuremath{\defaultFirstRelation,\defaultSecondRelation}}
\providecommandx*{\defaultNRelation}[2][1=1,2=n]{\ensuremath{\defaultRelation[#1],\ldots{},\defaultRelation[#2]}}
%% === Algebra ===
\providecommandx*{\defaultAdd}[1][1=\defaultSet]{\ensuremath{\defaultRelation[#1][+]}}
\providecommandx*{\defaultTimes}[1][1=\defaultSet]{\ensuremath{\defaultRelation[#1][\cdot]}}
\providecommandx*{\defaultZero}[1][1=\defaultSet]{\ensuremath{{0}_{#1}}}
\providecommandx*{\defaultUnit}[1][1=\defaultSet]{\ensuremath{{1}_{#1}}}
\providecommandx*{\Algebra}[2][1=\defaultSet,2=\defaultRelation]{\ensuremath{\left({#1,#2}\right)}}
\providecommandx*{\AddAlgebra}[1][1=\defaultSet]{\ensuremath{\Algebra[#1][{\defaultAdd[#1]}]}}
\providecommandx*{\TimesAlgebra}[1][1=\defaultSet]{\ensuremath{\Algebra[#1][{\defaultTimes[#1]}]}}
\providecommandx*{\GeneralAlgebra}[2][1=\defaultSet,2=\defaultNRelation]{\ensuremath{\Algebra[#1][#2]}}
\providecommandx*{\Op}[3][1=\defaultRelation]{\ensuremath{#2#1#3}}
%% === Group ===
\providecommandx*{\GroupSet}[1][1=G]{\ensuremath{#1}}
\providecommandx*{\GroupRelation}[2][1=\GroupSet,2=\cdot]{\ensuremath{#2_{#1}}}
\providecommandx*{\Group}[2][1=\GroupSet,2=\GroupRelation]{\ensuremath{\Algebra[#1][{#2[#1]}]}}
\providecommandx*{\GOp}[3][1=\GroupRelation]{\ensuremath{\Op[#1]{#2}{#3}}}
\fi

\chapter{代數結構}
\section{簡介}

\begin{mydef}[二元運算]
\label{def:algebra:binary_operation}
一個函數 $\Func{\defaultRelation}{A\times{B}}{C}$,對於所有 $a\in{A}$、$b\in{B}$,存在\textbf{唯一}的 $c\in{C}$,使得 $\defaultRelation{(a,b)}=c$,我們稱 $\defaultRelation$ 是一個從 $A\times{B}$ 到 $C$ 的\textbf{二元運算 (Binary Operation)},此時記為 $a\defaultRelation{b}=c$。
\end{mydef}
\begin{mynote*}
若 $A=B=C=\defaultSet$,我們稱 $\defaultRelation$ 是定義在 $\defaultSet$ 上的二元運算。
\end{mynote*}

\begin{myexample}下列為二元運算:
\begin{enumerate}
\item 整數加法 $\Func{+}{\Int\times{\Int}}{\Int}$
\item 實數乘法 $\Func{\cdot}{\Real^2}{\Real}$
\item 實係數矩陣乘法 $\Func{\cdot}{\Mat[m][n]{\Real}\times\Mat[n][p]{\Real}}{\Mat[m][p]{\Real}}$
\item 充要條件 $\Func{\Leftrightarrow}{{\LogicSet}\times{\LogicSet}}{\left\{{\True,\False}\right\}}$
\item $\mathcal{O}{(n^2)}$ 的 LCS 演算法 $\Func{\textsc{LCS}}{{{\{0,1\}}^{m}}\times{{\{0,1\}}^{n}}}{{\{0,1\}}^{k}}$
\end{enumerate}
\end{myexample}

\begin{mydef}[n元運算]
\label{def:algebra:n_ary_operation}
一個函數 $\defaultRelation: \NSet\rightarrow{B}$,對於所有 ${\left({\NSet[a][,]}\right)}\in{\NSet}$,存在\textbf{唯一}的 $b\in{B}$,使得 $\defaultRelation{\left({\NSet[a][,]}\right)}=b$,我們稱 $\defaultRelation$ 是一個從 $\NSet$ 到 $B$ 的\textbf{ n 元運算 (n-ary Operation)}。
\end{mydef}

\begin{mydef}[代數結構與代數系統]
\label{def:algebra:algebraic_structure}
一\textbf{代數結構 (Algebraic Structure)} $\GeneralAlgebra$ 滿足以下條件
\begin{enumerate}
\item 有一非空集合 $\defaultSet$
\item $\defaultNRelation$ 為定義在 $\defaultSet$ 上的二元運算
\item 一系列的公理 $\mathcal{A}$
\end{enumerate}
若 $\defaultNRelation$ 為定義在 $\defaultSet$ 上的 n 元運算,則稱 $\GeneralAlgebra$ 為\textbf{代數系統 (Algebraic System)}。
\end{mydef}

\begin{myexample}下列為代數結構:
\begin{enumerate}
\item 有理數與加法、乘法 $\Algebra[\Ration][+,\cdot]$
\item 複係數矩陣乘法 $\Algebra[\Mat{\Comp}][\cdot]$
\item 正整數與最大公因數 $\Algebra[\PosInt][\gcd]$,其中最大公因數為二元運算 $\Func{\gcd}{\PosInt\times{\PosInt}}{\PosInt}$
\item 函數合成 $\Algebra[\FuncSet][\circ]$
\end{enumerate}
\end{myexample}

\section{二元運算的性質}
\subsection{基本性質}

\begin{mydef}[封閉律]
\label{def:algebra:closure}
一個代數結構 $\Algebra$ 中,若對於所有 $a,b\in{\defaultSet}$,使得 $\Op{a}{b}\in{S}$,則稱二元運算 $\defaultRelation$ 對 $\defaultSet$ 滿足\textbf{封閉律 (Closure)}。
\end{mydef}

\begin{mydef}[結合律]
\label{def:algebra:associativity}
一個封閉的代數結構 $\Algebra$ 中,若對於所有 $a,b,c\in{\defaultSet}$,使得 $\Op{(\Op{a}{b})}{c}=\Op{a}{(\Op{b}{c})}$,則稱二元運算 $\defaultRelation$ 對 $\defaultSet$ 具有\textbf{結合律 (Associativity, Associative property)}。
\end{mydef}

\begin{mydef}[交換律]
\label{def:algebra:commutativity}
一個封閉的代數結構 $\Algebra$ 中,若對於所有 $a,b\in{\defaultSet}$,使得 $\Op{a}{b}=\Op{b}{a}$,則稱二元運算 $\defaultRelation$ 對 $\defaultSet$ 具有\textbf{交換律 (Commutativity, Commutative property)}。
\end{mydef}

\begin{myexample}
說明下列代數結構是否有交換律。
\begin{enumerate}
\item $\Algebra[\NonNegInt]$,$\forall{a,b}\in{\NonNegInt}$,$\Op{a}{b}=a^b$
\end{enumerate}
\end{myexample}
\begin{proof}
\begin{enumerate}
\item[]
\item 計算 $\Op{3}{2}=3^2=9$、$\Op{2}{3}=2^3=8$,因為 $9\neq{8}$,因此 $\Op{3}{2}\neq{\Op{2}{3}}$,$\defaultRelation$ 不具交換律。
\end{enumerate}
\end{proof}

\begin{myexample}
說明下列代數結構是否滿足封閉律。
\begin{enumerate}
\item $\Algebra[\Real][+]$
\item $\Algebra[\NonNegInt][/]$
\item $\Algebra[{\Int[\sqrt{2}]}][\cdot]$
\item $\Algebra[\Mat{\Comp}][\cdot]$
\item $\Algebra[\Irration][+]$
\item $\Algebra[\Real^2][\spadesuit]$,定義 $\Func{\spadesuit}{\Real\times{\Real^2}}{\Real^2}$,規則為:對所有 $a\in{\Real}$、${(x,y)}\in{\Real^2}$,使得 $a\spadesuit{(x,y)}=(x+a,y-a)$
\end{enumerate}
\end{myexample}
\begin{proof}
\begin{enumerate}
\item[]
\item[3.] 我們取任意 $a_1,a_2,b_1,b_2\in{\Int}$,計算
\begin{align*}
  & {({a_1+a_2\sqrt{2}})}\cdot{({b_1+b_2\sqrt{2}})}\\
= & (a_1b_1+2a_2b_2)+{(a_1b_2+a_2b_1)}\sqrt{2}
\end{align*}
發現 $a_1b_1+2a_2b_2\in{\Int}$ 且 $a_1b_2+a_2b_1\in{\Int}$,因此 $(a_1b_1+2a_2b_2)+{(a_1b_2+a_2b_1)}\sqrt{2}\in{\Int[\sqrt{2}]}$,因此 $\cdot$ 在 $\Int[\sqrt{2}]$ 中滿足封閉律。
\item[5.] 令 $1+\sqrt{2},1-\sqrt{2}\in{\Irration}$,我們發現 ${(1+\sqrt{2})}+{(1-\sqrt{2})}=2\notin{\Irration}$,因此 $\Algebra[\Irration][+]$ 不具封閉律。
\end{enumerate}
\end{proof}

\begin{mydef}[吸收律]
\label{def:algebra:absorption_law}
一個封閉的代數結構 $\Algebra[\defaultSet][\defaultTwoRelation]$ 中,若對於所有 $a,b\in{\defaultSet}$,使得
\begin{align*}
\Op[\defaultFirstRelation]{a}{({\Op[\defaultSecondRelation]{a}{b})}}=a\\
\Op[\defaultSecondRelation]{a}{({\Op[\defaultFirstRelation]{a}{b})}}=a
\end{align*}
,則稱二元運算 $\defaultTwoRelation$ 在 $\defaultSet$ 上滿足\textbf{吸收律 (Absorption law)}。
\end{mydef}
\begin{mynote*}
吸收律是定義在\textbf{一對}二元運算上,因此不能單獨定義一個運算子具有吸收律。
\end{mynote*}
\begin{myexample}
一個邏輯語句與 or 運算、and 運算 $(\LogicSet,\LogicJoin,\LogicMeet)$ 有吸收律,因為對於 $a$
\end{myexample}

\begin{mydef}[分配律]
\label{def:algebra:distributivity}
一個封閉的代數結構 $\Algebra[\defaultSet][\defaultTwoRelation]$ 中,若對於所有 $a,b,c\in{\defaultSet}$,使得
\begin{align*}
\Op[\defaultFirstRelation]{a}{({\Op[\defaultSecondRelation]{b}{c})}}=\Op[\defaultSecondRelation]{({\Op[\defaultFirstRelation]{a}{b}})}{({\Op[\defaultFirstRelation]{a}{c}})}\\
\Op[\defaultFirstRelation]{({\Op[\defaultSecondRelation]{b}{c})}}{a}=\Op[\defaultSecondRelation]{({\Op[\defaultFirstRelation]{b}{a}})}{({\Op[\defaultFirstRelation]{c}{a}})}
\end{align*}
,則稱二元運算 $\defaultFirstRelation$ 在 $\defaultSet$ 上對 $\defaultSecondRelation$ 具有\textbf{分配律 (Distributivity, Distributive property)}。
\end{mydef}
\begin{mynote*}
儘管 $\defaultFirstRelation$ 對 $\defaultSecondRelation$ 有分配律,但 $\defaultSecondRelation$ \textbf{未必}對 $\defaultFirstRelation$ 有分配律。
\end{mynote*}

\begin{myexample}
整數加法與乘法 $(\Int,+,\cdot)$ 中乘法對加法有分配律,加法對乘法則無。
\begin{align*}
2\cdot{(3+4)}&={(2\cdot{3})}+{(2\cdot{4})}\\
2+{(3\cdot{4})}&\neq{(2+3)\cdot{(2+4)}}
\end{align*}
\end{myexample}

\subsection{單位元素}

\begin{mydef}[單位元素]
\label{def:algebra:identity}
一個封閉的代數結構 $\Algebra$ 中,若
\begin{itemize}
\item 存在 $e_l\in{\defaultSet}$,對所有 $a\in{\defaultSet}$,$\Op{e_l}{a}=a$,則 $e_l$ 為\textbf{左單位元素 (Left identity)}
\item 存在 $e_r\in{\defaultSet}$,對所有 $a\in{\defaultSet}$,$\Op{a}{e_r}=a$,則 $e_r$ 為\textbf{右單位元素 (Right identity)}
\item 存在 $e\in{\defaultSet}$,對所有 $a\in{\defaultSet}$,$\Op{e}{a}=\Op{a}{e}=a$,則 $e$ 為\textbf{單位元素 (Identity)}
\end{itemize}
\end{mydef}

\begin{mythm}[單位元素存在性]
\label{thm:algebra:identity_existence}
一個封閉的代數結構 $\Algebra$ 中,若存在左單位元素 $e_l$、右單位元素 $e_r$,則 $e_l=e_r$,即單位元素存在。
\end{mythm}
\begin{proof}
根據定義 \ref{def:algebra:identity},我們知道對於所有元素 $a\in{\defaultSet}$,$\Op{e_l}{a}=a$ 且 $\Op{a}{e_r}=a$,我們嘗試去計算 $\Op{e_l}{e_r}$,因為 $e_l$ 是左單位元素,因此
\begin{align*}
\Op{e_l}{e_r}=e_r
\end{align*}
又因為 $e_r$ 是右單位元素,因此
\begin{align*}
\Op{e_l}{e_r}=e_l
\end{align*}
我們得到
\begin{align*}
e_l=\Op{e_l}{e_r}=e_r
\end{align*}
根據定義 \ref{def:algebra:identity},我們知道有一個單位元素即是 $e=e_l=e_r$ (因為左單位元素和右單位元素是同一個)。
\end{proof}

\begin{mythm}[單位元素唯一性]
\label{thm:algebra:identity_uniqueness}
一個封閉的代數結構 $\Algebra$ 中,若存在單位元素,則單位元素唯一。
\end{mythm}
\begin{proof}
不失一般性假設有兩個單位元素 $e_1$ 和 $e_2$,我們同樣下去計算 $\Op{e_1}{e_2}$,因為 $e_1$ 是單位元素,所以
\begin{align*}
\Op{e_1}{e_2}=e_2
\end{align*}
同時,$e_2$ 也是單位元素,因此
\begin{align*}
\Op{e_1}{e_2}=e_1
\end{align*}
我們得到
\begin{align*}
e_1=\Op{e_1}{e_2}=e_2
\end{align*}
\end{proof}

\subsection{反元素}

\begin{mydef}[反元素]
\label{def:algebra:inverse}
一個封閉的代數結構 $\Algebra$ 中,若
\begin{itemize}
\item 存在左單位元素 $e_l\in{\defaultSet}$,使得 $a,b_l\in{\defaultSet}$,$\Op{b_l}{a}=e_l$,則 $b_l$ 稱為 $a$ 的\textbf{左反元素 (Left inverse)}
\item 存在右單位元素 $e_r\in{\defaultSet}$,使得 $a,b_r\in{\defaultSet}$,$\Op{a}{b_r}=e_r$,則 $b_r$ 稱為 $a$ 的\textbf{右反元素 (Right inverse)}
\item 存在單位元素 $e\in{\defaultSet}$,使得 $a,b\in{\defaultSet}$,$\Op{b}{a}=\Op{a}{b}=e$,則 $b$ 稱為 $a$ 的\textbf{反元素 (Inverse)},$a$ 又稱\textbf{可逆元素 (Invertible element)}
\end{itemize}
若對所有 $a\in{\defaultSet}$ 都有反元素,則稱 $\defaultRelation$ 在 $\defaultSet$ 上有\textbf{反元素 (Inverse property)}。
\end{mydef}

\begin{mypropo}
\label{pro:algebra:identity_has_inverse}
一個封閉的代數結構 $\Algebra$ 存在單位元素 $e$,則 $e$ 的反元素為 $e$。
\end{mypropo}
\begin{proof}
根據定義 \ref{def:algebra:identity},我們知道 $\Op{e}{e}=e$,同時也符合反元素的定義。
\end{proof}

\begin{mythm}[反元素存在性]
\label{thm:algebra:inverse_existence}
\label{exe:algebra:inverse_existence}
一個封閉的代數結構 $\Algebra$ 存在單位元素 $e$,且 $\defaultRelation$ 具有結合律,若 $a\in{\defaultSet}$ 存在左反元素 $b_l$,右反元素 $b_r$,則 $b_l=b_r$,即反元素存在。
\end{mythm}
\begin{proof}
做為習題。
\end{proof}

\begin{mythm}[反元素唯一性]
\label{thm:algebra:inverse_uniqueness}
\label{exe:algebra:inverse_uniqueness}
一個封閉的代數結構 $\Algebra$ 存在單位元素 $e$,且 $\defaultRelation$ 具有結合律,若 $a\in{\defaultSet}$ 存在反元素,則反元素唯一。
\end{mythm}
\begin{proof}
做為習題。
\end{proof}

\begin{mydef}[反元素通用記號]
\label{def:algebra:inverse_general_notation}
一個封閉的代數結構 $\Algebra$ 中,若對 $a\in{\defaultSet}$ 有反元素,則此反元素記為 $a^{'}$。
\end{mydef}

\begin{mypropo}[反元素性質]
\label{pro:algebra:inverse_property}
一個封閉的代數結構 $\Algebra$ 若滿足以下條件:
\begin{itemize}
\item 有結合律
\item 有單位元素 $e$
\item 對所有 $a\in{\defaultSet}$ 都有反元素 $a^{'}$
\end{itemize}
則對所有 $a,b\in{\defaultSet}$ 有以下特性:
\begin{enumerate}
\item ${\left({a^{'}}\right)}^{'}=a$
\item \label{exe:algebra:inverse_property} ${\left({\Op{a}{b}}\right)}^{'}=\Op{b^{'}}{a^{'}}$
\end{enumerate}
\end{mypropo}
\begin{proof}
\begin{enumerate}
\item 依照定義,我們知道 $a^{'}$ 是 $a$ 的反元素、${\left({a^{'}}\right)}^{'}$ 是 $a^{'}$ 的反元素,因此
\begin{align*}
\Op{a^{'}}{a}&=\Op{a}{a^{'}}=e\\
\Op{{({a^{'}})}^{'}}{a^{'}}&=\Op{a^{'}}{{({a^{'}})}^{'}}=e
\end{align*}
則我們可以推導出:
\begin{align*}
{({a^{'}})}^{'} &= \Op{{({a^{'}})}^{'}}{e}               &e\text{ 是單位元素}\\
                &= \Op{{({a^{'}})}^{'}}{(\Op{a^{'}}{a})} &\Op{a^{'}}{a}=e\\
                &= \Op{(\Op{{({a^{'}})}^{'}}{a^{'}})}{a} &\text{結合律}\\
                &= \Op{e}{a}                             &\Op{{({a^{'}})}^{'}}{a^{'}}=e\\
                &= a                                     &e\text{ 是單位元素}
\end{align*}
\item 做為習題。
\end{enumerate}
\end{proof}

\begin{mythm}
\label{thm:algebra:left_identity_inverse}
一個封閉的代數結構 $\Algebra$ 若滿足結合律,則以下兩個敘述是等價的:
\begin{enumerate}
\item \label{thm:algebra:left_id_inv_first}
    \begin{enumerate}
    \item 有左單位元素 $e_l$
    \item 對所有 $a\in{\defaultSet}$,存在左反元素
    \end{enumerate}
\item \label{thm:algebra:left_id_inv_second}
    \begin{enumerate}
    \item $e_l$ 是單位元素
    \item \label{exe:algebra:left_identity_inverse}對所有 $a\in{\defaultSet}$,存在反元素
    \end{enumerate}
\end{enumerate}
\end{mythm}
\begin{proof}
我們要證明第 \ref{thm:algebra:left_id_inv_first} 項和第 \ref{thm:algebra:left_id_inv_second} 項等價,因此我們有兩部分要證明:第一、證明第 \ref{thm:algebra:left_id_inv_first} 項可以推到第 \ref{thm:algebra:left_id_inv_second} 項;第二、證明第 \ref{thm:algebra:left_id_inv_second} 項可以推到第 \ref{thm:algebra:left_id_inv_first} 項。
\begin{enumerate}
\item 我們先證第 \ref{thm:algebra:left_id_inv_second} 項推到第 \ref{thm:algebra:left_id_inv_first} 項 ($\Leftarrow$):
    \begin{enumerate}
    \item 根據定義 \ref{def:algebra:identity},我們有單位元素 $e_l$,換句話說 $e_l$ 也是左單位元素。
    \item 同樣地,根據定義 \ref{def:algebra:inverse},我們馬上就可以得到對於所有 $a\in{\defaultSet}$,存在左反元素。
    \end{enumerate}
\item 再證第 \ref{thm:algebra:left_id_inv_first} 項可以推到第 \ref{thm:algebra:left_id_inv_second} 項 ($\Rightarrow$):
    \begin{enumerate}
    \item 根據定義 \ref{def:algebra:identity},我們證明 $e_l$ 是單位元素,只要證明對所有 $a\in{\defaultSet}$,都符合 $\Op{a}{e_l}=a$ 即可。根據定義 \ref{def:algebra:inverse},假設 $b_l$ 是 $a$ 的左反元素,我們有
    \begin{align*}
    \Op{b_l}{a}=e_l
    \end{align*}
    假設 $d_l$ 是 $b_l$ 的左反元素,我們也可得到:
    \begin{align*}
    \Op{d_l}{b_l}=e_l
    \end{align*}
    接著我們計算 $\Op{a}{e_l}$:
    \begin{align*}
    \Op{a}{e_l} &= \Op{e_l}{(\Op{a}{e_l})}             &e_l\text{ 是 }(\Op{a}{e_l})\text{ 的左單位元素}\\
                &= \Op{(\Op{d_l}{b_l})}{(\Op{a}{e_l})} &\text{因為 }\Op{d_l}{b_l}=e_l\\
                &= \Op{d_l}{(\Op{b_l}{(\Op{a}{e_l})})} &\text{結合律}\\
                &= \Op{d_l}{(\Op{(\Op{b_l}{a})}{e_l})} &\text{結合律}\\
                &= \Op{d_l}{(\Op{e_l}{e_l})}           &\text{因為 }\Op{b_l}{a}=e_l\\
                &= \Op{d_l}{e_l}                       &e_l\text{ 是 }e_l\text{ 的左單位元素}\\
                &= \Op{d_l}{(\Op{b_l}{a})}             &\text{因為 }\Op{b_l}{a}=e_l\\
                &= \Op{(\Op{d_l}{b_l})}{a}             &\text{結合律}\\
                &= \Op{e_l}{a}                         &\text{因為 }\Op{d_l}{b_l}=e_l\\
                &= a                                   &e_l\text{ 是 }a\text{ 的左單位元素}
    \end{align*}
    得出對所有 $a\in{\defaultSet}$,使得 $\Op{e_l}{a}=\Op{a}{e_l}=e_l$,因此 $e_l$ 是單位元素。
    \item 做為習題。
    \end{enumerate}
\end{enumerate}
\end{proof}
\begin{mynote*}
若是只有右單位元素 $e_r$,以及對所有 $a\in{\defaultSet}$ 有右反元素 $b_r$ 的時候,也會有類似的性質。
\end{mynote*}

\subsection{零元素與零因子}

\begin{mydef}[零元素]
\label{def:algebra:zero_element}
一個封閉的代數結構 $\Algebra$ 中,若
\begin{itemize}
\item 存在 $z_l\in{\defaultSet}$,對所有 $a\in{\defaultSet}$,$\Op{z_l}{a}=z_l$,則 $z_l$ 為\textbf{左零元素 (Left zero element)}
\item 存在 $z_r\in{\defaultSet}$,對所有 $a\in{\defaultSet}$,$\Op{a}{z_r}=z_r$,則 $z_r$ 為\textbf{右零元素 (Right zero element)}
\item 存在 $z\in{\defaultSet}$,對所有 $a\in{\defaultSet}$,$\Op{z}{a}=\Op{a}{z}=z$,則 $z$ 為\textbf{零元素 (Zero element)}
\end{itemize}
\end{mydef}
\begin{mynote*}
零元素又稱\textbf{吸收元素 (Absorbing element)}。
\end{mynote*}

\begin{mythm}[零元素存在性]
\label{thm:algebra:zero_existence}
\label{exe:algebra:zero_existence}
一個封閉的代數結構 $\Algebra$ 具有結合律,若存在左零元素 $z_l$,右零元素 $z_r$,則 $z_l=z_r$,即零元素存在。
\end{mythm}
\begin{proof}
做為習題。
\end{proof}

\begin{mythm}[零元素唯一性]
\label{thm:algebra:zero_uniqueness}
\label{exe:algebra:zero_uniqueness}
一個封閉的代數結構 $\Algebra$ 具有結合律,若存在零元素,則零元素唯一。
\end{mythm}
\begin{proof}
做為習題。
\end{proof}

\begin{mythm}
\label{thm:algebra:identity_zero_distinct}
一個封閉的代數結構 $\Algebra$ 有單位元素 $e$、零元素 $z$,若 $|\defaultSet|\geq{2}$,則 $e\neq{z}$。
\end{mythm}
\begin{proof}
我們用反證法證明,假設 $e=z$,則對於所有 $a\in{\defaultSet}$,我們發現
\begin{align*}
a &= \Op{a}{e} &e\text{ 是單位元素}\\
  &= \Op{a}{z} &e=z\\
  &= z         &z\text{ 是零元素}\\
  &= e         &e=z
\end{align*}
我們求出所有的 $a=e=z$ 都是相同的元素,因此 $|\defaultSet|=1$,與 $|\defaultSet|\geq{2}$ 矛盾。
\end{proof}

\begin{mypropo}
\label{pro:algebra:zero_no_inverse}
\label{exe:algebra:zero_no_inverse}
一個封閉的代數結構 $\Algebra$ 存在單位元素 $e$,若 $\defaultRelation$ 在 $\defaultSet$ 上有零元素 $z$ 且 $z\neq{e}$,則 $z$ 沒有反元素。
\end{mypropo}
\begin{proof}
做為習題。
\end{proof}

\begin{mydef}[零因子]
\label{def:algebra:zero_divisor}
一個封閉的代數結構 $\Algebra$ 中存在零元素 $z$,若 $a,b\in{\defaultSet}$ 且 $a,b\neq{z}$,使得 $\Op{a}{b}=z$,則 $a,b$ 稱為\textbf{零因子 (Zero divisor)}。
\end{mydef}

\begin{mythm}[零因子性質]
\label{thm:algebra:zero_divisor_no_inverse}
一個封閉的代數結構 $\Algebra$ 滿足以下條件:
\begin{itemize}
\item 有結合律
\item 存在單位元素 $e$
\item 存在零元素 $z$
\end{itemize}
若 $a,b\in{\defaultSet}$ 是零因子,則 $a,b$ 沒有反元素。
\end{mythm}
\begin{proof}
因為 $a,b$ 是零因子,所以 $\Op{a}{b}=z$ 且 $a,b\neq{z}$。先假設 $a$ 有反元素 $a^{'}\in{\defaultSet}$,我們知道
\begin{align*}
z &= \Op{a^{'}}{z}           &z\text{ 是零元素,因此 }\Op{a^{'}}{z}=z\\
  &= \Op{a^{'}}{(\Op{a}{b})} &\Op{a}{b}=z\\
  &= \Op{(\Op{a^{'}}{a})}{b} &\text{結合律}\\
  &= \Op{e}{b}               &a^{'}\text{ 是 }a\text{ 的反元素,因此 }\Op{a^{'}}{a}=e\\
  &= b                       &e\text{ 是單位元素}
\end{align*}
我們求出 $z=b$,與原來的前提 ($a,b\neq{z}$) 矛盾。同理,$b$ 也沒有反元素。
\end{proof}

\begin{mydef}[消去律]
\label{def:algebra:cancellation_law}
一個封閉的代數結構 $\Algebra$ 中,對所有 $a,b,c\in{\defaultSet}$,若
\begin{itemize}
\item $\Op{a}{b}=\Op{a}{c}$ 可得到 $b=c$,則 $\defaultRelation$ 在 $\defaultSet$ 上有\textbf{左消去律 (Left cancellation law)}。
\item $\Op{b}{a}=\Op{c}{a}$ 可得到 $b=c$,則 $\defaultRelation$ 在 $\defaultSet$ 上有\textbf{右消去律 (Right cancellation law)}。
\item $\defaultRelation$ 滿足左消去律和右消去律,則 $\defaultRelation$ 在 $\defaultSet$ 上有\textbf{消去律 (Cancellation law)}。
\end{itemize}
\end{mydef}

\begin{mythm}[消去律性質]
\label{thm:algebra:cancellation_law}
\label{exe:algebra:cancellation_law}
一個封閉的代數結構 $\Algebra$ 若滿足以下條件:
\begin{itemize}
\item 有結合律
\item 有單位元素 $e$
\item 對所有 $a\in{\defaultSet}$ 都有反元素
\end{itemize}
則 $\defaultRelation$ 在 $\defaultSet$ 上有消去律。
\end{mythm}
\begin{proof}
根據定義 \ref{def:algebra:cancellation_law},我們要驗證 $\defaultRelation$ 有左消去律和右消去律。
\paragraph{左消去律}對於所有 $a,b,c\in{\defaultSet}$,驗證 $\Op{a}{b}=\Op{a}{c}$ 是否能推導出 $b=c$。假設 $a$ 有反元素 $a^{'}\in{\defaultSet}$,則
\begin{align*}
\Op{a}{b}=\Op{a}{c} &\Rightarrow \Op{a^{'}}{(\Op{a}{b})}=\Op{a^{'}}{(\Op{a}{c})} &\text{等式兩邊同時與 }a^{'}\text{ 做運算}\\
                    &\Rightarrow \Op{(\Op{a^{'}}{a})}{b}=\Op{(\Op{a^{'}}{a})}{c} &\text{結合律}\\
                    &\Rightarrow \Op{e}{b}=\Op{e}{c}                             &a^{'}\text{ 是 }a\text{ 的反元素}\\
                    &\Rightarrow b=c                                             &e\text{ 是單位元素}
\end{align*}
\paragraph{右消去律}做為習題。
\end{proof}

\begin{mythm}
\label{thm:algebra:identity_inverse_equilibrium}
\label{exe:algebra:identity_inverse_equilibrium}
一個封閉的代數結構 $\Algebra$ 若滿足結合律,則以下兩個敘述是等價的:
\begin{enumerate}
\item \label{thm:algebra:id_inv_eq_first}
    \begin{enumerate}
    \item $e$ 是單位元素
    \item 對所有 $a\in{\defaultSet}$,存在反元素
    \end{enumerate}
\item \label{thm:algebra:id_inv_eq_second} 對於任意 $a,b\in{\defaultSet}$,$x,y$ 是在 $\defaultSet$ 的未知數,方程式 $\Op{a}{x}=b$ 和 $\Op{y}{a}=b$ 存在唯一解。
\end{enumerate}
\end{mythm}
\begin{proof}
\begin{enumerate}
\item ($\Rightarrow$) 方向:已知有單位元素 $e$ 和反元素,我們要證存在性和唯一性。
    \begin{enumerate}
    \item 先證存在性,我們只要找出一組解就可以證明存在性。假設 $a$ 有反元素 $a^{'}\in{\defaultSet}$,我們可以找出當 $x=\Op{a^{'}}{b}$ 時,
    \begin{align*}
    \Op{a}{x} &= \Op{a}{(\Op{a^{'}}{b})} &x=\Op{a^{'}}{b}\\
              &= \Op{(\Op{a}{a^{'}})}{b} &\text{結合律}\\
              &= \Op{e}{b}               &a^{'}\text{ 是 }a\text{ 的反元素}\\
              &= b                       &e\text{ 是單位元素}
    \end{align*}
    \item 再證唯一性,假設 $x$ 有兩個解 $c$ 和 $d$,亦即 $\Op{a}{c}=b=\Op{a}{d}$,則
    \begin{align*}
    c &= \Op{e}{c}               &e\text{ 是單位元素}\\
      &= \Op{(\Op{a^{'}}{a})}{c} &\Op{a^{'}}{a}=e\\
      &= \Op{a^{'}}{(\Op{a}{c})} &\text{結合律}\\
      &= \Op{a^{'}}{(\Op{a}{d})} &\Op{a}{c}=b=\Op{a}{d}\\
      &= \Op{(\Op{a^{'}}{a})}{d} &\text{結合律}\\
      &= \Op{e}{d}               &\Op{a^{'}}{a}=e\\
      &= d                       &e\text{ 是單位元素}
    \end{align*}
    \item 同理,也可證明 $y$ 存在唯一解 $\Op{b}{c}$。
    \end{enumerate}
\item ($\Leftarrow$) 方向:做為習題。
\end{enumerate}
\end{proof}

\subsection{符號簡化}

\newcommand{\opRelation}{\ensuremath{\mathfrak{R}}}
\newcommand{\opR}{\mathop{\vphantom{\sum}\mathchoice
  {\vcenter{\hbox{\huge \opRelation}}}
  {\vcenter{\hbox{\Large \opRelation}}}
  {\mathrm{\opRelation}}
  {\mathrm{\opRelation}}
}\displaylimits}

\begin{mydef}[連運算記號]
\label{def:algebra:accumulating_relational_notation}
一個封閉的代數結構 $\Algebra$ 中,對於任意 $a_1,\cdots,a_n\in{\defaultSet}$,定義運算 $\opR^{n}_{i=1}{a_i}$,其中 $n\in{\PosInt}$:
\begin{enumerate}
\item 當 $n=1$ 時,$\opR^{n}_{i=1}{a_i}=a_1$
\item 當 $n>1$ 時,$\opR^{n}_{i=1}{a_i}=\Op{a_1}{(\Op{a_2}{(\Op{\cdots}{a_n})})}=\Op{a_1}{(\opR^{n}_{i=2}{a_i})}$
\end{enumerate}
\end{mydef}

\begin{mypropo}[連運算性質]
\label{pro:algebra:accumulating_relational_notation}
\label{exe:algebra:accumulating_relational_notation}
一個封閉的代數結構 $\Algebra$ 中若滿足結合律,則對於 $n,m\in{\PosInt}$、$n<m$ 且 $a_1,\cdots,a_{n+m}\in{\defaultSet}$ 滿足
\begin{align*}
\Op{(\opR^{n}_{i=1}{a_i})}{(\opR^{n+m}_{j=n+1}{a_j})}=\opR^{n+m}_{k=1}{a_k}
\end{align*}
\end{mypropo}
\begin{proof}
做為習題。
\end{proof}

\begin{mydef}
\label{def:algebra:general_associativity}
一個封閉的代數結構 $\Algebra$ 中,對於任意 $a_1,\cdots,a_n\in{\defaultSet}$ 做運算,且這 $n$ 個元素不得改變次序,所有以某個順序運算 $a_1,\cdots,a_n$ 的集合,我們記為 $\Phi(a_1,\cdots,a_n)$。
\end{mydef}
\begin{myexample}
假設 $a_1,a_2,a_3,a_4\in{\defaultSet}$,則 $\Phi(a_1,a_2,a_3,a_4)$ 有 5 個元素:
\begin{enumerate}
\item $\Op{a_1}{(\Op{a_2}{(\Op{a_3}{a_4})})}$
\item $\Op{a_1}{(\Op{(\Op{a_2}{a_3})}{a_4})}$
\item $\Op{(\Op{a_1}{a_2})}{(\Op{a_3}{a_4})}$
\item $\Op{(\Op{a_1}{(\Op{a_2}{a_3})})}{a_4}$
\item $\Op{(\Op{(\Op{a_1}{a_2})}{a_3})}{a_4}$
\end{enumerate}
\end{myexample}

\begin{mythm}[廣義結合律]
\label{thm:algebra:general_associativity}
一個封閉的代數結構 $\Algebra$ 中若滿足結合律,對於任意 $a_1,\cdots,a_n\in{\defaultSet}$,則所有 $\phi\in{\Phi(a_1,\cdots,a_n)}$,$\phi=\opR^{n}_{i=1}{a_i}$。
\end{mythm}
\begin{proof}
我們用強數學歸納法證明:
\begin{enumerate}
\item 先證明 $k=1$,對所有 $\phi\in{\Phi(a_1)=\{a_1\}}$,$\phi=a_1=\opR^{k}_{i=1}{a_i}$
\item 假設對所有 $1\leq{k}\leq{n}$ 都滿足此性質,接著證明當 $k=n+1$ 時,對於所有 $\phi\in{\Phi(a_1,\cdots,a_n)}$,因為 $\defaultRelation$ 是二元運算,最後必有 $\phi_{l}\in{\Phi(a_1,\cdots,a_x)}$ 且 $\phi_{r}\in{\Phi(a_{x+1},\cdots,a_{n+1})}$,$1\leq{x}\leq{n+1}$,使得
\begin{align*}
\phi &= \Op{\phi_{l}}{\phi_{r}}                               &\\
     &= \Op{(\opR^{x}_{i=1}{a_i})}{(\opR^{n+1}_{j=x+1}{a_j})} &\text{強數學歸納法假設}\\
     &= \opR^{n+1}_{i=1}{a_i}                                 &\text{根據性質 \ref{pro:algebra:accumulating_relational_notation}}
\end{align*}
\end{enumerate}
\end{proof}
\begin{mynote*}
也就是說,一封閉的代數結構 $\Algebra$ 有結合律,則 $a_1,\cdots,a_n\in{\defaultSet}$ 只要前後次序不變,則結果相同。
\end{mynote*}

\begin{mydef}[倍運算記號]
\label{def:algebra:multiplex_relational_notation}
一個封閉的代數結構 $\Algebra$ 中若滿足以下條件:
\begin{itemize}
\item 結合律
\item 有單位元素 $e$
\end{itemize}
則對於任意 $a\in{\defaultSet}$,定義運算 $\opR^{n}{a}$,其中 $n\in{\PosInt}$:
\begin{enumerate}
\item 當 $n=1$,$\opR^{n}{a}=\opR^{1}{a}=a$
\item 當 $n>1$,$\opR^{n}{a}=\Op{a}{(\opR^{n-1}{a})}$
\end{enumerate}
\end{mydef}
\begin{mynote*}
根據廣義結合律的性質 \ref{pro:algebra:accumulating_relational_notation},無論這幾個 $a$ 以何種順序運算,結果皆相同。
\end{mynote*}

\begin{mydef}[單位元素記號]
\label{def:algebra:identity_notation}
一個封閉的代數結構 $\Algebra$ 存在單位元素 $e$,若
\begin{itemize}
\item $\defaultRelation$ 為加法 $\defaultAdd$,則 $e$ 為\textbf{加法單位元素 (Additive identity)},此時 $e$ 記為 $\defaultZero$。
\item $\defaultRelation$ 為乘法 $\defaultTimes$,則 $e$ 為\textbf{乘法單位元素 (Multiplicative identity)},$e$ 記為 $\defaultUnit$。
\end{itemize}
\end{mydef}
\begin{mynote*}
\begin{enumerate}
\item $\defaultAdd$ 不是真的代表實數或複數的加法運算,而是代表他在 $\defaultSet$ 上有類似我們常見的加法性質,因此用這個符號容易聯想;$\defaultTimes$ 亦然。
\item 使用 $0$ 和 $1$ 做為記號只是方便我們去聯想他的性質,事實上並不是實數的「$0$」和「$1$」,只是單純的\textbf{符號}。
\end{enumerate}
\end{mynote*}

\begin{mydef}[反元素記號]
\label{def:algebra:inverse_notation}
一個封閉的代數結構 $\Algebra$ 存在單位元素 $e$,且所有 $a\in{\defaultSet}$ 均有反元素 $a^{'}\in{\defaultSet}$,若
\begin{itemize}
\item $\defaultRelation$ 為加法 $\defaultAdd$,則 $a^{'}$ 為\textbf{加法反元素 (Additive inverse)},此時 $a^{'}$ 記為 $-a$。
\item $\defaultRelation$ 為乘法 $\defaultTimes$,則 $a^{'}$ 為\textbf{乘法反元素 (Multiplicative inverse)},此時 $a^{'}$ 記為 $a^{-1}$。
\end{itemize}
\end{mydef}
\begin{mynote*}
同樣地,$-a$ 和 $a^{-1}$ 只是單純的符號,不要和減法與倒數搞混。
\end{mynote*}

\subsection{其他性質}

\begin{mydef}[冪等元素與冪等律]
\label{def:algebra:idempotent}
一個封閉的代數結構 $\Algebra$ 中,若有 $a\in{\defaultSet}$,使得 $\Op{a}{a}=a$,則 $a$ 稱為\textbf{冪等元素 (Idempotent element)}。若所有 $a\in{\defaultSet}$ 都是冪等元素,則稱二元運算 $\defaultRelation$ 在 $\defaultSet$ 上滿足\textbf{冪等律 (Idempotent)}。
\end{mydef}

\begin{mydef}[冪零元素與冪零律]
\label{def:algebra:nilpotent}
一個封閉的代數結構 $\Algebra$ 中存在零元素 $z$,若 $a\in{\defaultSet}$,使得 $\opR^{k}{a}=z$,$k\in{\PosInt}$,則 $a$ 稱為\textbf{冪零元素 (Nilpotent element)}。則稱二元運算 $\defaultRelation$ 對 $\defaultSet$ 滿足\textbf{冪零律 (Nilpotent)}。
\end{mydef}

\section{同態與同構}

\begin{mydef}[代數結構分類]
兩個代數結構 $\Algebra[\defaultSet][{\defaultNRelation[\defaultSet,1][\defaultSet,n]}]$、$\Algebra[T][{\defaultNRelation[T,1][T,n]}]$,若有相同的公理 $\mathcal{A}$,則我們稱 $\defaultSet$ 和 $T$ 是同類代數結構。
\end{mydef}

\begin{mydef}[同態]
\label{def:algebra:homomorphism}
兩個同類的代數結構 $\Algebra[\defaultSet][{\defaultNRelation[\defaultSet,1][\defaultSet,n]}]$、$\Algebra[T][{\defaultNRelation[T,1][T,n]}]$ 若能找到一函數 $\Func{f}{\defaultSet}{T}$,使得對所有 $a,b\in{\defaultSet}$、$1\leq{i}\leq{n}$ 都遵守
\begin{align*}
{f(\Op[{\defaultRelation[\defaultSet,i]}]{a}{b})}=\Op[{\defaultRelation[T,i]}]{f(a)}{f(b)}
\end{align*}
則稱 $\defaultSet$ 和 $T$ \textbf{同態 (Homomorphism)}。
\end{mydef}

\begin{mydef}[同構]
\label{def:algebra:isomorphism}
兩個同類的代數結構 $\Algebra[\defaultSet][{\defaultNRelation[\defaultSet,1][\defaultSet,n]}]$、$\Algebra[T][{\defaultNRelation[T,1][T,n]}]$ 若能找到一函數 $\Func{f}{\defaultSet}{T}$ 滿足以下條件
\begin{enumerate}
\item $\defaultSet$ 和 $T$ 同態
\item $f$ 是雙射函數
\end{enumerate}
則稱 $\defaultSet$ 和 $T$ \textbf{同構 (Isomorphism)},記為 $S\cong{T}$,$f$ 稱為同構函數。
\end{mydef}

\section*{習題}

\begin{enumerate}
\item 定義一個在 $\Int$ 上二元運算 $\diamondsuit$,對所有 $x,y\in{\Int}$,使得 $\Op[\diamondsuit]{x}{y}=3x+y-4$。問 ${({(7\diamondsuit{5})}\diamondsuit{3})}-{7\diamondsuit{(5\diamondsuit{3})}}$?
\item 證明定理 \ref{exe:algebra:inverse_existence}。
\item 證明定理 \ref{exe:algebra:inverse_uniqueness}。
\item 證明定理 \ref{thm:algebra:left_identity_inverse} 第 \ref{exe:algebra:left_identity_inverse} 項。
\item 證明性質 \ref{pro:algebra:inverse_property} 第 \ref{exe:algebra:inverse_property} 項。
\item 證明定理 \ref{exe:algebra:zero_existence}。
\item 證明定理 \ref{exe:algebra:zero_uniqueness}。
\item 證明性質 \ref{exe:algebra:zero_no_inverse}。
\item 證明定理 \ref{exe:algebra:cancellation_law} 右消去律部分。
\item 證明定理 \ref{exe:algebra:identity_inverse_equilibrium} ($\Leftarrow$) 部分。
\item 證明性質 \ref{exe:algebra:accumulating_relational_notation}。
\item 一封閉的代數結構 $\Algebra$ 有交換律,試證明:
\begin{enumerate}
\item 若 $\defaultRelation$ 有左單位元素 $e_l$,則單位元素存在
\item 對所有 $a\in{\defaultSet}$ 都有左反元素 $b_l$,則反元素存在
\end{enumerate}
\item 一個封閉的代數結構 $\Algebra$ 滿足以下條件:
\begin{itemize}
\item 有結合律
\item 有左單位元素 $e_l$
\item 對所有 $a\in{\defaultSet}$ 都有左反元素
\end{itemize}
則 $\defaultRelation$ 在 $\defaultSet$ 上有左消去律。
\end{enumerate}

\ifx \allfiles \undefined
\printindex[noun]

\clearpage
\end{CJK}
\end{document}
\fi
\part{群論}
\ifx \allfiles \undefined
\documentclass[12pt,a4paper,oneside]{report}

%% === CJK 套件 ===
\usepackage{CJKutf8,CJKnumb}                    % 中文套件
\usepackage[unicode]{hyperref,xcolor}
\hypersetup{
    colorlinks,
    linkcolor={blue!100!black},
    citecolor={blue!75!black},
    urlcolor={blue!50!black}
}
%% === AMS 標準套件 ===
\usepackage{amsmath,amsfonts,amssymb,amsthm}    % 數學符號
%% === 章節內容 ===
\usepackage{enumitem}                           % 修改 enumerate, item
\usepackage{titletoc,titlesec}                  % titletoc 目錄修改套件, titlesec 美化章節標題套件
\usepackage{imakeidx}                           % 索引
%% ===  ===
\usepackage[chapter]{algorithm}                 % 演算法套件
\usepackage[noend]{algpseudocode}               % pseudocode 套件
\usepackage{listings}                           % 程式碼
%% ===  ===
\usepackage{tikz,tkz-graph,tkz-berge}
%% ===  ===
\usepackage{xkeyval,xargs}

\makeindex[name=noun]        % 索引生成
\linespread{1.24}

%% === 設定頁面格式 ===
%\hoffset         = 10pt                      % 水平位移,預設為 0pt
\voffset         = -15pt                     % 垂直位移,預設為 0pt
\oddsidemargin   = 0pt                       % 預設為 31pt
%\topmargin       = 20pt                      % 預設為 20pt
%\headheight      = 12pt                      % header 的高度,預設為 12pt
%\headsep         = 25pt                      % header 和 body 的距離,預設為 25pt
\textheight      = 620pt                     % body 內文部分的高度,預設為 592pt
\textwidth       = 450pt                     % body 內文部分的寬度,預設為 390pt
%\marginparsep    = 10pt                      % margin note 和 body 的距離,預設為 10pt
%\marginparwidth  = 35pt                      % margin note 的寬度,預設為 35pt
%\footskip        = 30pt                      % footer 高度 + footer 和 body 的距離,預設為 30pt

%% === itemize,enumerate 設定 ===
%  使用 enumitem 套件
\setlist[itemize]{itemsep=0pt,parsep=0pt}
\setlist[enumerate]{itemsep=0pt,parsep=0pt}

\begin{document}
\begin{CJK}{UTF8}{bkai}
\newtheorem{mydef}{定義}[chapter]
\newtheorem*{mydef*}{定義}
\newtheorem{myrule}[mydef]{原理}
\newtheorem{mythm}[mydef]{定理}
\newtheorem{mylma}[mydef]{引理}
\newtheorem{mypropo}[mydef]{性質}
\newtheorem{mycorol}[mydef]{推論}
\newtheorem{myexample}[mydef]{範例}
\newtheorem*{mynote*}{註}
\numberwithin{equation}{section}
\renewenvironment{proof}{\textbf{證明}}{\qed}
\newenvironment{mysol}{\textbf{解答}}{\qed}

%% === 載入符號表 ===
%% === Basic Number Symbols ===
\providecommandx*{\PosInt}{\ensuremath{{\mathbb{Z}^{+}}}}
\providecommandx*{\NonNegInt}{\ensuremath{\mathbb{N}}}
\providecommandx*{\Int}{\ensuremath{\mathbb{Z}}}
\providecommandx*{\Ration}{\ensuremath{\mathbb{Q}}}
\providecommandx*{\Irration}{\ensuremath{\mathbb{Q}_{c}}}
\providecommandx*{\Real}{\ensuremath{\mathbb{R}}}
\providecommandx*{\Comp}{\ensuremath{\mathbb{C}}}
\providecommandx*{\Mat}[3][1=n,2=n]{\ensuremath{\mathbb{M}_{#1\times{#2}}{\left({#3}\right)}}}
\providecommandx*{\FuncSet}[3][1=\mathbb{F},2=\mathbb{R},3=\mathbb{R}]{\ensuremath{#1{\left({#2,#3}\right)}}}
%% ===  ===
\providecommandx*{\True}{\ensuremath{\top}}
\providecommandx*{\False}{\ensuremath{\bot}}
%% ===  ===
\providecommandx*{\Func}[3]{\ensuremath{#1:#2\rightarrow{#3}}}
\providecommandx*{\DefSet}[2]{\ensuremath{\left\{{#1|#2}\right\}}}
\providecommandx*{\NSet}[3][1=A,2=\times,3=n]{\ensuremath{{#1}_{1}#2{{#1}_{2}}#2\ldots{#2{#1}_{#3}}}}

%% ===  ===
\providecommandx*{\defaultSet}{\ensuremath{S}}
\providecommandx*{\defaultRelation}[2][1=,2=\mathcal{R}]{\ensuremath{{#2}_{#1}}}
\providecommandx*{\defaultFirstRelation}{\ensuremath{\defaultRelation[1]}}
\providecommandx*{\defaultSecondRelation}{\ensuremath{\defaultRelation[2]}}
\providecommandx*{\defaultTwoRelation}{\ensuremath{\defaultFirstRelation,\defaultSecondRelation}}
\providecommandx*{\defaultNRelation}[2][1=1,2=n]{\ensuremath{\defaultRelation[#1],\ldots{},\defaultRelation[#2]}}
%% === Algebra ===
\providecommandx*{\defaultAdd}[1][1=\defaultSet]{\ensuremath{\defaultRelation[#1][+]}}
\providecommandx*{\defaultTimes}[1][1=\defaultSet]{\ensuremath{\defaultRelation[#1][\cdot]}}
\providecommandx*{\defaultZero}[1][1=\defaultSet]{\ensuremath{{0}_{#1}}}
\providecommandx*{\defaultUnit}[1][1=\defaultSet]{\ensuremath{{1}_{#1}}}
\providecommandx*{\Algebra}[2][1=\defaultSet,2=\defaultRelation]{\ensuremath{\left({#1,#2}\right)}}
\providecommandx*{\AddAlgebra}[1][1=\defaultSet]{\ensuremath{\Algebra[#1][{\defaultAdd[#1]}]}}
\providecommandx*{\TimesAlgebra}[1][1=\defaultSet]{\ensuremath{\Algebra[#1][{\defaultTimes[#1]}]}}
\providecommandx*{\GeneralAlgebra}[2][1=\defaultSet,2=\defaultNRelation]{\ensuremath{\Algebra[#1][#2]}}
\providecommandx*{\Op}[3][1=\defaultRelation]{\ensuremath{#2#1#3}}
%% === Group ===
\providecommandx*{\GroupSet}[1][1=G]{\ensuremath{#1}}
\providecommandx*{\GroupRelation}[2][1=\GroupSet,2=\cdot]{\ensuremath{#2_{#1}}}
\providecommandx*{\Group}[2][1=\GroupSet,2=\GroupRelation]{\ensuremath{\Algebra[#1][{#2[#1]}]}}
\providecommandx*{\GOp}[3][1=\GroupRelation]{\ensuremath{\Op[#1]{#2}{#3}}}
\fi

\chapter{群}
\section{定義與性質}

\begin{mydef}[群]
\label{def:group:group}
一個代數結構 $\Group$ 被稱為\textbf{群 (Group)},滿足以下條件:
\begin{enumerate}[label=(G\arabic*),labelindent=\parindent,leftmargin=*]
\item \label{def:group:g1} 有封閉律,對於所有 $a,b\in{\GroupSet}$,$\GOp{a}{b}\in{\GroupSet}$
\item \label{def:group:g2} 有結合律,對於所有 $a,b,c\in{\GroupSet}$,$\GOp{(\GOp{a}{b})}{c}=\GOp{a}{(\GOp{b}{c})}$
\item \label{def:group:g3} 有單位元素 $e\in{\GroupSet}$,使得所有 $a\in{\GroupSet}$,$\GOp{e}{a}=\GOp{a}{e}=a$
\item \label{def:group:g4} 對於每個元素 $a\in{\GroupSet}$ 都有反元素 $a^{'}\in{\GroupSet}$,使得 $\GOp{a}{a^{'}}=\GOp{a^{'}}{a}=e$
\end{enumerate}
此時 $\GroupRelation$ 稱為\textbf{群乘法}。
\end{mydef}

\begin{mypropo}[群的性質]
\label{pro_group_identity_inverse}
若 $\Group$ 為一個群,則有以下性質:
\begin{enumerate}
\item 單位元素唯一。
\item 對於所有 $a\in{\GroupSet}$,其反元素唯一。
\end{enumerate}
\end{mypropo}
\begin{proof}
\begin{enumerate}
\item 根據定理 \ref{thm:algebra:identity_uniqueness} 得證。
\item 根據定理 \ref{thm:algebra:inverse_uniqueness} 得證。
\end{enumerate}
\end{proof}

\begin{mydef}[符號簡化]
\label{def:group:symbols_simplify}
若一個群 $\GroupSet$ 的二元運算為群乘法 $\GroupRelation$,則我們可對符號簡化:
\begin{enumerate}
\item 對於所有 $a,b\in{\GroupSet}$,$\GOp{a}{b}$ 可寫為 $ab$。
\item 對於所有 $a\in{\GroupSet}$,其反元素記為 $a^{-1}$。
\end{enumerate}
\end{mydef}

\begin{mypropo}
\label{def:group:inverse_property}
若 $\Group$ 為一個群,對於所有 $a,b\in{\GroupSet}$,則有以下性質:
\begin{enumerate}
\item ${\left({a^{-1}}\right)}^{-1}=a$
\item ${\left({ab}\right)}^{-1}=b^{-1}a^{-1}$
\end{enumerate}
\end{mypropo}
\begin{mynote*}
${\left({a^{-1}}\right)}^{-1}$ 應理解為「$a^{-1}$ 的反元素」。
\end{mynote*}
\begin{proof}
\begin{enumerate}
\item 我們知道 $a^{-1}$ 是 $a$ 的反元素,且 ${\left({a^{-1}}\right)}^{-1}$ 也是 $a^{-1}$ 的反元素,根據群的定義 \ref{def:group:g4},我們知道
\begin{align*}
a^{-1}a&=aa^{-1}=e\\
{\left({a^{-1}}\right)}^{-1}a^{-1}&=a^{-1}{\left({a^{-1}}\right)}^{-1}=e
\end{align*}
因此
\begin{align*}
a &= ea                                    &\text{群的定義 \ref{def:group:g3}}\\
  &= ({\left({a^{-1}}\right)}^{-1}a^{-1})a &{\left({a^{-1}}\right)}^{-1}a^{-1}=e\\
  &= {\left({a^{-1}}\right)}^{-1}(a^{-1}a) &\text{群的定義 \ref{def:group:g2}}\\
  &= {\left({a^{-1}}\right)}^{-1}e         &a^{-1}a=e\\
  &= {\left({a^{-1}}\right)}^{-1}          &\text{群的定義 \ref{def:group:g3}}
\end{align*}
\item 做為習題。
\end{enumerate}
\end{proof}

\begin{mydef}[群連乘]
\label{def:group:group_multiplication}
若一個群 $\GroupSet$ 的二元運算為群乘法 $\GroupRelation$,則定義群連乘 $a^k$,$k\in{\Int}$:
\begin{enumerate}
\item $k=0$ 時,$a^k=a^0=e$
\item $k>0$ 時,$a^k=\GOp{a}{a^{k-1}}=aa^{k-1}$;$a^{-k}={(a^{-1})}^k$
\end{enumerate}
\end{mydef}

\begin{mycorol}[群連乘性質]
一個群 $\GroupSet$ 中,對所有 $a\in{\GroupSet}$,$m,n\in{\Int}$,則
\begin{enumerate}
\item ${a^m}{a^n}=a^{m+n}={a^n}{a^m}$
\item ${({a^m})}^n=a^{mn}$
\item ${({a^m})}^{-1}=a^{-m}$
\end{enumerate}
\end{mycorol}

\begin{mypropo}[群的消去律]
\label{pro:group:cancellation_law}
若 $\Group$ 為一個群,則 $\GroupSet$ 滿足消去律。即對於所有 $a,b,c\in{\GroupSet}$,若
\begin{enumerate}
\item $ab=ac$,則 $b=c$
\item $ba=ca$,則 $b=c$
\end{enumerate}
\end{mypropo}
\begin{proof}
根據定理 \ref{thm:algebra:cancellation_law} 得證。
\end{proof}

\begin{mythm}
\label{thm:group:equilibrium}
若 $\Group$ 為一個群,則對於任意 $a,b\in{\GroupSet}$,$x,y$ 是未知數,$ax=b$ 和 $ya=b$ 存在唯一解。
\end{mythm}
\begin{proof}
由定理 \ref{thm:algebra:identity_inverse_equilibrium} 可知唯一解為 $x=a^{-1}b$ 且 $y=ba^{-1}$。
\end{proof}

\begin{mydef}[阿貝爾群]
\label{def:group:abelian_group}
一個群 $\Group$,若對所有 $a,b\in{\GroupSet}$ 都滿足 $ab=ba$,則 $\GroupSet$ 稱為\textbf{阿貝爾群 (Abelian group)},又稱\textbf{交換群}。
\end{mydef}
\begin{myexample}
$\Algebra[\Int][+]$ 是一個交換群,$\Algebra[\Mat{\Real}][\cdot]$ 就不是,反例:
\begin{align*}
A=\left(\begin{array}{cc}
1 & 0\\
0 & 0
\end{array}\right),
B=\left(\begin{array}{cc}
0 & 1\\
0 & 0
\end{array}\right)\\
AB=\left(\begin{array}{cc}
0 & 1\\
0 & 0
\end{array}\right)\neq\left(\begin{array}{cc}
0 & 0\\
0 & 0
\end{array}\right)=BA
\end{align*}
。
\end{myexample}

\begin{mydef}[半群和單半群]
\label{def:group:semigroup_and_monoid}
一個代數結構 $\Group$ 若滿足 \ref{def:group:g1} 和 \ref{def:group:g2},則 $\GroupSet$ 稱為\textbf{半群 (Semigroup)}。若 $\Group$ 滿足 \ref{def:group:g1}、\ref{def:group:g2}、\ref{def:group:g3},則 $\GroupSet$ 稱為\textbf{單半群 (Monoid)}。
\end{mydef}

\begin{mydef}[有限群]
\label{def:group:finite_group}
一個群 $\Group$ 若 $|G|<\infty$,則 $\GroupSet$ 稱為\textbf{有限群 (Finite group)}。
\end{mydef}

\section{子群}

\begin{mydef}[子群]
\label{def:group:subgroup}
一個群 $\Group$ 上,若 $\SubGroup$ 是一個群且 $\SubGroupSet\subseteq{\GroupSet}$、$\SubGroupSet\neq{\emptyset}$,則 $\SubGroup$ 被稱為 $G$ 的\textbf{子群 (Subgroup)}。
\end{mydef}
\begin{mynote*}
$\GroupRelation[\GroupSet]$ 和 $\GroupRelation[\SubGroupSet]$ 要是相同的。
\end{mynote*}

\begin{mylma}[實用版子群]
\label{lma:group:two_rule_subgroup}
一個群 $\Group$ 上有一個非空子集 $\SubGroupSet$,$\SubGroupSet$ 滿足以下條件
\begin{enumerate}
\item 對於任意 $a,b\in{\SubGroupSet}$,$ab\in{\SubGroupSet}$
\item 對於任意 $a\in{\SubGroupSet}$,存在 $a^{-1}\in{\SubGroupSet}$
\end{enumerate}
若且唯若 $\SubGroup$ 是一個子群。
\end{mylma}
\begin{mynote*}
事實上這兩個規則就是驗證 \ref{def:group:g1} 和 \ref{def:group:g4}。
\end{mynote*}

\begin{mylma}[精簡版子群]
\label{lma:group:one_rule_subgroup}
一個群 $\Group$ 上有一個非空子集 $\SubGroupSet$,對於任意 $a,b\in{\SubGroupSet}$,$ab^{-1}\in{\SubGroupSet}$ 若且唯若 $\SubGroup$ 是一個子群。
\end{mylma}

\begin{mypropo}
\label{pro:group:two_subgroup_intersection}
一個群 $\Group$ 上若有兩個子群 $\SubGroupSet_1$、$\SubGroupSet_2$,則 ${\SubGroupSet_1}\cap{\SubGroupSet_2}$ 是 $\GroupSet$ 的子群。
\end{mypropo}

\begin{mythm}[有限子群]
一個有限群 $\Group$ 上有一個非空子集 $\SubGroupSet$,對於任意 $a,b\in{\SubGroupSet}$,$ab\in{\SubGroupSet}$ 若且唯若 $\SubGroup$ 是一個子群。
\end{mythm}

\ifx \allfiles \undefined
\printindex[noun]

\clearpage
\end{CJK}
\end{document}
\fi
%\printindex[noun]
%\listoffigures
\clearpage
\end{CJK}
\end{document}