\ifx \allfiles \undefined
\documentclass[12pt,a4paper,oneside]{report}

%% === CJK 套件 ===
\usepackage{CJKutf8,CJKnumb}                    % 中文套件
\usepackage[unicode]{hyperref,xcolor}
\hypersetup{
    colorlinks,
    linkcolor={blue!100!black},
    citecolor={blue!75!black},
    urlcolor={blue!50!black}
}
%% === AMS 標準套件 ===
\usepackage{amsmath,amsfonts,amssymb,amsthm}    % 數學符號
%% === 章節內容 ===
\usepackage{enumitem}                           % 修改 enumerate, item
\usepackage{titletoc,titlesec}                  % titletoc 目錄修改套件, titlesec 美化章節標題套件
\usepackage{imakeidx}                           % 索引
%% ===  ===
\usepackage[chapter]{algorithm}                 % 演算法套件
\usepackage[noend]{algpseudocode}               % pseudocode 套件
\usepackage{listings}                           % 程式碼
%% ===  ===
\usepackage{tikz,tkz-graph,tkz-berge}
%% ===  ===
\usepackage{xkeyval,xargs}

\makeindex[name=noun]        % 索引生成
\linespread{1.24}

%% === 設定頁面格式 ===
%\hoffset         = 10pt                      % 水平位移,預設為 0pt
\voffset         = -15pt                     % 垂直位移,預設為 0pt
\oddsidemargin   = 0pt                       % 預設為 31pt
%\topmargin       = 20pt                      % 預設為 20pt
%\headheight      = 12pt                      % header 的高度,預設為 12pt
%\headsep         = 25pt                      % header 和 body 的距離,預設為 25pt
\textheight      = 620pt                     % body 內文部分的高度,預設為 592pt
\textwidth       = 450pt                     % body 內文部分的寬度,預設為 390pt
%\marginparsep    = 10pt                      % margin note 和 body 的距離,預設為 10pt
%\marginparwidth  = 35pt                      % margin note 的寬度,預設為 35pt
%\footskip        = 30pt                      % footer 高度 + footer 和 body 的距離,預設為 30pt

%% === itemize,enumerate 設定 ===
%  使用 enumitem 套件
\setlist[itemize]{itemsep=0pt,parsep=0pt}
\setlist[enumerate]{itemsep=0pt,parsep=0pt}

\begin{document}
\begin{CJK}{UTF8}{bkai}
\newtheorem{mydef}{定義}[chapter]
\newtheorem*{mydef*}{定義}
\newtheorem{myrule}[mydef]{原理}
\newtheorem{mythm}[mydef]{定理}
\newtheorem{mylma}[mydef]{引理}
\newtheorem{mypropo}[mydef]{性質}
\newtheorem{mycorol}[mydef]{推論}
\newtheorem{myexample}[mydef]{範例}
\newtheorem*{mynote*}{註}
\numberwithin{equation}{section}
\renewenvironment{proof}{\textbf{證明}}{\qed}
\newenvironment{mysol}{\textbf{解答}}{\qed}

%% === 載入符號表 ===
%% === Basic Number Symbols ===
\providecommandx*{\PosInt}{\ensuremath{{\mathbb{Z}^{+}}}}
\providecommandx*{\NonNegInt}{\ensuremath{\mathbb{N}}}
\providecommandx*{\Int}{\ensuremath{\mathbb{Z}}}
\providecommandx*{\Ration}{\ensuremath{\mathbb{Q}}}
\providecommandx*{\Irration}{\ensuremath{\mathbb{Q}_{c}}}
\providecommandx*{\Real}{\ensuremath{\mathbb{R}}}
\providecommandx*{\Comp}{\ensuremath{\mathbb{C}}}
\providecommandx*{\Mat}[3][1=n,2=n]{\ensuremath{\mathbb{M}_{#1\times{#2}}{\left({#3}\right)}}}
\providecommandx*{\FuncSet}[3][1=\mathbb{F},2=\mathbb{R},3=\mathbb{R}]{\ensuremath{#1{\left({#2,#3}\right)}}}
%% ===  ===
\providecommandx*{\True}{\ensuremath{\top}}
\providecommandx*{\False}{\ensuremath{\bot}}
%% ===  ===
\providecommandx*{\Func}[3]{\ensuremath{#1:#2\rightarrow{#3}}}
\providecommandx*{\DefSet}[2]{\ensuremath{\left\{{#1|#2}\right\}}}
\providecommandx*{\NSet}[3][1=A,2=\times,3=n]{\ensuremath{{#1}_{1}#2{{#1}_{2}}#2\ldots{#2{#1}_{#3}}}}

%% ===  ===
\providecommandx*{\defaultSet}{\ensuremath{S}}
\providecommandx*{\defaultRelation}[2][1=,2=\mathcal{R}]{\ensuremath{{#2}_{#1}}}
\providecommandx*{\defaultFirstRelation}{\ensuremath{\defaultRelation[1]}}
\providecommandx*{\defaultSecondRelation}{\ensuremath{\defaultRelation[2]}}
\providecommandx*{\defaultTwoRelation}{\ensuremath{\defaultFirstRelation,\defaultSecondRelation}}
\providecommandx*{\defaultNRelation}[2][1=1,2=n]{\ensuremath{\defaultRelation[#1],\ldots{},\defaultRelation[#2]}}
%% === Algebra ===
\providecommandx*{\defaultAdd}[1][1=\defaultSet]{\ensuremath{\defaultRelation[#1][+]}}
\providecommandx*{\defaultTimes}[1][1=\defaultSet]{\ensuremath{\defaultRelation[#1][\cdot]}}
\providecommandx*{\defaultZero}[1][1=\defaultSet]{\ensuremath{{0}_{#1}}}
\providecommandx*{\defaultUnit}[1][1=\defaultSet]{\ensuremath{{1}_{#1}}}
\providecommandx*{\Algebra}[2][1=\defaultSet,2=\defaultRelation]{\ensuremath{\left({#1,#2}\right)}}
\providecommandx*{\AddAlgebra}[1][1=\defaultSet]{\ensuremath{\Algebra[#1][{\defaultAdd[#1]}]}}
\providecommandx*{\TimesAlgebra}[1][1=\defaultSet]{\ensuremath{\Algebra[#1][{\defaultTimes[#1]}]}}
\providecommandx*{\GeneralAlgebra}[2][1=\defaultSet,2=\defaultNRelation]{\ensuremath{\Algebra[#1][#2]}}
\providecommandx*{\Op}[3][1=\defaultRelation]{\ensuremath{#2#1#3}}
%% === Group ===
\providecommandx*{\GroupSet}[1][1=G]{\ensuremath{#1}}
\providecommandx*{\GroupRelation}[2][1=\GroupSet,2=\cdot]{\ensuremath{#2_{#1}}}
\providecommandx*{\Group}[2][1=\GroupSet,2=\GroupRelation]{\ensuremath{\Algebra[#1][{#2[#1]}]}}
\providecommandx*{\GOp}[3][1=\GroupRelation]{\ensuremath{\Op[#1]{#2}{#3}}}
\fi

\chapter{群}
\section{定義與性質}

\begin{mydef}[群]
\label{def:group:group}
一個代數結構 $\Group$ 被稱為\textbf{群 (Group)},滿足以下條件:
\begin{enumerate}[label=(G\arabic*),labelindent=\parindent,leftmargin=*]
\item \label{def:group:g1} 有封閉律,對於所有 $a,b\in{\GroupSet}$,$\GOp{a}{b}\in{\GroupSet}$
\item \label{def:group:g2} 有結合律,對於所有 $a,b,c\in{\GroupSet}$,$\GOp{(\GOp{a}{b})}{c}=\GOp{a}{(\GOp{b}{c})}$
\item \label{def:group:g3} 有單位元素 $e\in{\GroupSet}$,使得所有 $a\in{\GroupSet}$,$\GOp{e}{a}=\GOp{a}{e}=a$
\item \label{def:group:g4} 對於每個元素 $a\in{\GroupSet}$ 都有反元素 $a^{'}\in{\GroupSet}$,使得 $\GOp{a}{a^{'}}=\GOp{a^{'}}{a}=e$
\end{enumerate}
此時 $\GroupRelation$ 稱為\textbf{群乘法}。
\end{mydef}

\begin{mypropo}[群的性質]
\label{pro_group_identity_inverse}
若 $\Group$ 為一個群,則有以下性質:
\begin{enumerate}
\item 單位元素唯一。
\item 對於所有 $a\in{\GroupSet}$,其反元素唯一。
\end{enumerate}
\end{mypropo}
\begin{proof}
\begin{enumerate}
\item 根據定理 \ref{thm:algebra:identity_uniqueness} 得證。
\item 根據定理 \ref{thm:algebra:inverse_uniqueness} 得證。
\end{enumerate}
\end{proof}

\begin{mydef}[符號簡化]
\label{def:group:symbols_simplify}
若一個群 $\GroupSet$ 的二元運算為群乘法 $\GroupRelation$,則我們可對符號簡化:
\begin{enumerate}
\item 對於所有 $a,b\in{\GroupSet}$,$\GOp{a}{b}$ 可寫為 $ab$。
\item 對於所有 $a\in{\GroupSet}$,其反元素記為 $a^{-1}$。
\end{enumerate}
\end{mydef}

\begin{mypropo}
\label{def:group:inverse_property}
若 $\Group$ 為一個群,對於所有 $a,b\in{\GroupSet}$,則有以下性質:
\begin{enumerate}
\item ${\left({a^{-1}}\right)}^{-1}=a$
\item ${\left({ab}\right)}^{-1}=b^{-1}a^{-1}$
\end{enumerate}
\end{mypropo}
\begin{mynote*}
${\left({a^{-1}}\right)}^{-1}$ 應理解為「$a^{-1}$ 的反元素」。
\end{mynote*}
\begin{proof}
\begin{enumerate}
\item 我們知道 $a^{-1}$ 是 $a$ 的反元素,且 ${\left({a^{-1}}\right)}^{-1}$ 也是 $a^{-1}$ 的反元素,根據群的定義 \ref{def:group:g4},我們知道
\begin{align*}
a^{-1}a&=aa^{-1}=e\\
{\left({a^{-1}}\right)}^{-1}a^{-1}&=a^{-1}{\left({a^{-1}}\right)}^{-1}=e
\end{align*}
因此
\begin{align*}
a &= ea                                    &\text{群的定義 \ref{def:group:g3}}\\
  &= ({\left({a^{-1}}\right)}^{-1}a^{-1})a &{\left({a^{-1}}\right)}^{-1}a^{-1}=e\\
  &= {\left({a^{-1}}\right)}^{-1}(a^{-1}a) &\text{群的定義 \ref{def:group:g2}}\\
  &= {\left({a^{-1}}\right)}^{-1}e         &a^{-1}a=e\\
  &= {\left({a^{-1}}\right)}^{-1}          &\text{群的定義 \ref{def:group:g3}}
\end{align*}
\item 做為習題。
\end{enumerate}
\end{proof}

\begin{mydef}[群連乘]
\label{def:group:group_multiplication}
若一個群 $\GroupSet$ 的二元運算為群乘法 $\GroupRelation$,則定義群連乘 $a^k$,$k\in{\Int}$:
\begin{enumerate}
\item $k=0$ 時,$a^k=a^0=e$
\item $k>0$ 時,$a^k=\GOp{a}{a^{k-1}}=aa^{k-1}$;$a^{-k}={(a^{-1})}^k$
\end{enumerate}
\end{mydef}

\begin{mycorol}[群連乘性質]
一個群 $\GroupSet$ 中,對所有 $a\in{\GroupSet}$,$m,n\in{\Int}$,則
\begin{enumerate}
\item ${a^m}{a^n}=a^{m+n}={a^n}{a^m}$
\item ${({a^m})}^n=a^{mn}$
\item ${({a^m})}^{-1}=a^{-m}$
\end{enumerate}
\end{mycorol}

\begin{mypropo}[群的消去律]
\label{pro:group:cancellation_law}
若 $\Group$ 為一個群,則 $\GroupSet$ 滿足消去律。即對於所有 $a,b,c\in{\GroupSet}$,若
\begin{enumerate}
\item $ab=ac$,則 $b=c$
\item $ba=ca$,則 $b=c$
\end{enumerate}
\end{mypropo}
\begin{proof}
根據定理 \ref{thm:algebra:cancellation_law} 得證。
\end{proof}

\begin{mythm}
\label{thm:group:equilibrium}
若 $\Group$ 為一個群,則對於任意 $a,b\in{\GroupSet}$,$x,y$ 是未知數,$ax=b$ 和 $ya=b$ 存在唯一解。
\end{mythm}
\begin{proof}
由定理 \ref{thm:algebra:identity_inverse_equilibrium} 可知唯一解為 $x=a^{-1}b$ 且 $y=ba^{-1}$。
\end{proof}

\begin{mydef}[阿貝爾群]
\label{def:group:abelian_group}
一個群 $\Group$,若對所有 $a,b\in{\GroupSet}$ 都滿足 $ab=ba$,則 $\GroupSet$ 稱為\textbf{阿貝爾群 (Abelian group)},又稱\textbf{交換群}。
\end{mydef}
\begin{myexample}
$\Algebra[\Int][+]$ 是一個交換群,$\Algebra[\Mat{\Real}][\cdot]$ 就不是,反例:
\begin{align*}
A=\left(\begin{array}{cc}
1 & 0\\
0 & 0
\end{array}\right),
B=\left(\begin{array}{cc}
0 & 1\\
0 & 0
\end{array}\right)\\
AB=\left(\begin{array}{cc}
0 & 1\\
0 & 0
\end{array}\right)\neq\left(\begin{array}{cc}
0 & 0\\
0 & 0
\end{array}\right)=BA
\end{align*}
\end{myexample}

\begin{mydef}[半群和單半群]
\label{def:group:semigroup_and_monoid}
一個代數結構 $\Group$ 若滿足 \ref{def:group:g1} 和 \ref{def:group:g2},則 $\GroupSet$ 稱為\textbf{半群 (Semigroup)}。若 $\Group$ 滿足 \ref{def:group:g1}、\ref{def:group:g2}、\ref{def:group:g3},則 $\GroupSet$ 稱為\textbf{單半群 (Monoid)}。
\end{mydef}

\begin{mydef}[有限群]
\label{def:group:finite_group}
一個群 $\Group$ 若 $|G|<\infty$,則 $\GroupSet$ 稱為\textbf{有限群 (Finite group)}。
\end{mydef}

\section{子群}

\begin{mydef}[子群]
\label{def:group:subgroup}
一個群 $\Group$ 上,若 $\SubGroup$ 是一個群且 $\SubGroupSet\subseteq{\GroupSet}$、$\SubGroupSet\neq{\emptyset}$,則 $\SubGroup$ 被稱為 $G$ 的\textbf{子群 (Subgroup)}。
\end{mydef}
\begin{mynote*}
$\GroupRelation[\GroupSet]$ 和 $\GroupRelation[\SubGroupSet]$ 要是相同的。
\end{mynote*}

\begin{mylma}[實用版子群]
\label{lma:group:two_rule_subgroup}
一個群 $\Group$ 上有一個非空子集 $\SubGroupSet$,$\SubGroupSet$ 滿足以下條件
\begin{enumerate}
\item 對於任意 $a,b\in{\SubGroupSet}$,$ab\in{\SubGroupSet}$
\item 對於任意 $a\in{\SubGroupSet}$,存在 $a^{-1}\in{\SubGroupSet}$
\end{enumerate}
若且唯若 $\SubGroup$ 是一個子群。
\end{mylma}
\begin{mynote*}
事實上這兩個規則就是驗證 \ref{def:group:g1} 和 \ref{def:group:g4}。
\end{mynote*}

\begin{mylma}[精簡版子群]
\label{lma:group:one_rule_subgroup}
一個群 $\Group$ 上有一個非空子集 $\SubGroupSet$,對於任意 $a,b\in{\SubGroupSet}$,$ab^{-1}\in{\SubGroupSet}$ 若且唯若 $\SubGroup$ 是一個子群。
\end{mylma}

\begin{mypropo}
\label{pro:group:two_subgroup_intersection}
一個群 $\Group$ 上若有兩個子群 $\SubGroupSet_1$、$\SubGroupSet_2$,則 ${\SubGroupSet_1}\cap{\SubGroupSet_2}$ 是 $\GroupSet$ 的子群。
\end{mypropo}

\begin{mythm}[有限子群]
一個有限群 $\Group$ 上有一個非空子集 $\SubGroupSet$,對於任意 $a,b\in{\SubGroupSet}$,$ab\in{\SubGroupSet}$ 若且唯若 $\SubGroup$ 是一個子群。
\end{mythm}

\ifx \allfiles \undefined
\printindex[noun]

\clearpage
\end{CJK}
\end{document}
\fi