\documentclass[utf8]{beamer}

\usepackage{CJKutf8}
\usepackage{amsmath,amsfonts,amssymb}
\usepackage{tkz-graph,tkz-berge}
\usepackage{multicol}
\usepackage{xkeyval,xargs}

\usetheme{Boadilla}
\usecolortheme{crane}

\setbeamertemplate{items}[circle]

\begin{document}
\begin{CJK}{UTF8}{bkai}

\newtheorem{mydef}{定義}[section]
\newtheorem*{mydef*}{定義}
\newtheorem{myrule}[mydef]{原理}
\newtheorem{mythm}[mydef]{定理}
\newtheorem{mylma}[mydef]{引理}
%\newtheorem{mypropo}[mydef]{性質}
%\newtheorem{mycorol}[mydef]{推論}
%\newtheorem{myexample}[mydef]{範例}
%\newtheorem*{mynote*}{註}
\numberwithin{equation}{section}
\renewenvironment{proof}{\textbf{證明}}{\qed}
\newenvironment{mysol}{\textbf{解答}}{\qed}
\newenvironment{mypropo}{\begin{exampleblock}{性質}}{\end{exampleblock}}
\newenvironment{mycorol}{\begin{exampleblock}{推論}}{\end{exampleblock}}
\newenvironment{myexample}{\begin{exampleblock}{範例}}{\end{exampleblock}}
\newenvironment{mynote*}{\begin{alertblock}{註}}{\end{alertblock}}

%% === 載入符號表 ===
%% === Basic Number Symbols ===
\providecommandx*{\PosInt}{\ensuremath{{\mathbb{Z}^{+}}}}
\providecommandx*{\NonNegInt}{\ensuremath{\mathbb{N}}}
\providecommandx*{\Int}{\ensuremath{\mathbb{Z}}}
\providecommandx*{\Ration}{\ensuremath{\mathbb{Q}}}
\providecommandx*{\Irration}{\ensuremath{\mathbb{Q}_{c}}}
\providecommandx*{\Real}{\ensuremath{\mathbb{R}}}
\providecommandx*{\Comp}{\ensuremath{\mathbb{C}}}
\providecommandx*{\Mat}[3][1=n,2=n]{\ensuremath{\mathbb{M}_{#1\times{#2}}{\left({#3}\right)}}}
\providecommandx*{\FuncSet}[3][1=\mathbb{F},2=\mathbb{R},3=\mathbb{R}]{\ensuremath{#1{\left({#2,#3}\right)}}}
%% ===  ===
\providecommandx*{\True}{\ensuremath{\top}}
\providecommandx*{\False}{\ensuremath{\bot}}
%% ===  ===
\providecommandx*{\Func}[3]{\ensuremath{#1:#2\rightarrow{#3}}}
\providecommandx*{\DefSet}[2]{\ensuremath{\left\{{#1|#2}\right\}}}
\providecommandx*{\NSet}[3][1=A,2=\times,3=n]{\ensuremath{{#1}_{1}#2{{#1}_{2}}#2\ldots{#2{#1}_{#3}}}}

%% ===  ===
\providecommandx*{\defaultSet}{\ensuremath{S}}
\providecommandx*{\defaultRelation}[2][1=,2=\mathcal{R}]{\ensuremath{{#2}_{#1}}}
\providecommandx*{\defaultFirstRelation}{\ensuremath{\defaultRelation[1]}}
\providecommandx*{\defaultSecondRelation}{\ensuremath{\defaultRelation[2]}}
\providecommandx*{\defaultTwoRelation}{\ensuremath{\defaultFirstRelation,\defaultSecondRelation}}
\providecommandx*{\defaultNRelation}[2][1=1,2=n]{\ensuremath{\defaultRelation[#1],\ldots{},\defaultRelation[#2]}}
%% === Algebra ===
\providecommandx*{\defaultAdd}[1][1=\defaultSet]{\ensuremath{\defaultRelation[#1][+]}}
\providecommandx*{\defaultTimes}[1][1=\defaultSet]{\ensuremath{\defaultRelation[#1][\cdot]}}
\providecommandx*{\defaultZero}[1][1=\defaultSet]{\ensuremath{{0}_{#1}}}
\providecommandx*{\defaultUnit}[1][1=\defaultSet]{\ensuremath{{1}_{#1}}}
\providecommandx*{\Algebra}[2][1=\defaultSet,2=\defaultRelation]{\ensuremath{\left({#1,#2}\right)}}
\providecommandx*{\AddAlgebra}[1][1=\defaultSet]{\ensuremath{\Algebra[#1][{\defaultAdd[#1]}]}}
\providecommandx*{\TimesAlgebra}[1][1=\defaultSet]{\ensuremath{\Algebra[#1][{\defaultTimes[#1]}]}}
\providecommandx*{\GeneralAlgebra}[2][1=\defaultSet,2=\defaultNRelation]{\ensuremath{\Algebra[#1][#2]}}
\providecommandx*{\Op}[3][1=\defaultRelation]{\ensuremath{#2#1#3}}
%% === Group ===
\providecommandx*{\GroupSet}[1][1=G]{\ensuremath{#1}}
\providecommandx*{\GroupRelation}[2][1=\GroupSet,2=\cdot]{\ensuremath{#2_{#1}}}
\providecommandx*{\Group}[2][1=\GroupSet,2=\GroupRelation]{\ensuremath{\Algebra[#1][{#2[#1]}]}}
\providecommandx*{\GOp}[3][1=\GroupRelation]{\ensuremath{\Op[#1]{#2}{#3}}}

\title{基礎抽象代數 群}
\author{許胖}
\institute[PCSH]{板燒高中}

\begin{frame}
  \titlepage
\end{frame}
\begin{frame}
  \frametitle{大綱}
  \tableofcontents
\end{frame}

\section{定義與性質}

\begin{frame}
  \frametitle{定義}
  \begin{mydef}[群]
  \label{def:group:group}
  一個代數結構 $\Group$ 被稱為\textbf{群 (Group)},滿足以下條件:
  \begin{enumerate}
  \item<2-> \label{def:group:g1} 有封閉律,對於所有 $a,b\in{\GroupSet}$,$\GOp{a}{b}\in{\GroupSet}$
  \item<3-> \label{def:group:g2} 有結合律,對於所有 $a,b,c\in{\GroupSet}$,$\GOp{(\GOp{a}{b})}{c}=\GOp{a}{(\GOp{b}{c})}$
  \item<4-> \label{def:group:g3} 有單位元素 $e\in{\GroupSet}$,使得所有 $a\in{\GroupSet}$,$\GOp{e}{a}=\GOp{a}{e}=a$
  \item<5-> \label{def:group:g4} 對於每個元素 $a\in{\GroupSet}$ 都有反元素 $a^{'}\in{\GroupSet}$,使得 $\GOp{a}{a^{'}}=\GOp{a^{'}}{a}=e$
  \end{enumerate}
  此時 $\GroupRelation$ 稱為\textbf{群乘法}。
  \end{mydef}
\end{frame}

\begin{frame}
  \frametitle{群的性質 (1)}
  \begin{mypropo}
  \label{pro_group_identity_inverse}
  若 $\Group$ 為一個群,則有以下性質:
  \begin{enumerate}
  \item<2-> 單位元素唯一。
  \item<3-> 對於所有 $a\in{\GroupSet}$,其反元素唯一。
  \end{enumerate}
  \end{mypropo}
  \begin{mydef}[符號簡化]<4->
  \label{def:group:symbols_simplify}
  若一個群 $\GroupSet$ 的二元運算為群乘法 $\GroupRelation$,則我們可對符號簡化:
  \begin{enumerate}
  \item 對於所有 $a,b\in{\GroupSet}$,$\GOp{a}{b}$ 可寫為 $ab$。
  \item 對於所有 $a\in{\GroupSet}$,其反元素記為 $a^{-1}$。
  \end{enumerate}
  \end{mydef}
\end{frame}

\begin{frame}
  \frametitle{群的性質 (2)}
  \begin{mypropo}
  \label{def:group:inverse_property}
  若 $\Group$ 為一個群,對於所有 $a,b\in{\GroupSet}$,則有以下性質:
  \begin{enumerate}
  \item<2-> ${\left({a^{-1}}\right)}^{-1}=a$
  \item<3-> ${\left({ab}\right)}^{-1}=b^{-1}a^{-1}$
  \end{enumerate}
  \end{mypropo}
  \begin{mynote*}<1->
  ${\left({a^{-1}}\right)}^{-1}$ 應理解為「$a^{-1}$ 的反元素」。
  \end{mynote*}
\end{frame}

\begin{frame}
  \frametitle{群連乘}
  \begin{mydef}[群連乘]
  \label{def:group:group_multiplication}
  若一個群 $\GroupSet$ 的二元運算為群乘法 $\GroupRelation$,則定義群連乘 $a^k$,$k\in{\Int}$:
  \begin{enumerate}
  \item $k=0$ 時,$a^k=a^0=e$
  \item $k>0$ 時,$a^k=\GOp{a}{a^{k-1}}=aa^{k-1}$;$a^{-k}={(a^{-1})}^k$
  \end{enumerate}
  \end{mydef}
  \begin{mycorol}<2->
  一個群 $\GroupSet$ 中,對所有 $a\in{\GroupSet}$,$m,n\in{\Int}$,則
  \begin{enumerate}
  \item<2-> ${a^m}{a^n}=a^{m+n}={a^n}{a^m}$
  \item<3-> ${({a^m})}^n=a^{mn}$
  \item<4-> ${({a^m})}^{-1}=a^{-m}$
  \end{enumerate}
  \end{mycorol}
\end{frame}

\begin{frame}
  \frametitle{群的消去律}
  \begin{mypropo}
  \label{pro:group:cancellation_law}
  若 $\Group$ 為一個群,則 $\GroupSet$ 滿足消去律。即對於所有 $a,b,c\in{\GroupSet}$,若
  \begin{enumerate}
  \item $ab=ac$,則 $b=c$
  \item $ba=ca$,則 $b=c$
  \end{enumerate}
  \end{mypropo}
  \begin{mythm}<2->
  \label{thm:group:equilibrium}
  若 $\Group$ 為一個群,則對於任意 $a,b\in{\GroupSet}$,$x,y$ 是未知數,$ax=b$ 和 $ya=b$ 存在唯一解。
  \end{mythm}
\end{frame}

\begin{frame}
  \frametitle{阿貝爾群}
  \begin{mydef}[阿貝爾群]
  \label{def:group:abelian_group}
  一個群 $\Group$,若對所有 $a,b\in{\GroupSet}$ 都滿足 $ab=ba$,則 $\GroupSet$ 稱為\textbf{阿貝爾群 (Abelian group)},又稱\textbf{交換群}。
  \end{mydef}
  \begin{myexample}
  $\Algebra[\Int][+]$ 是一個交換群,$\Algebra[\Mat{\Real}][\cdot]$ 就不是,反例:\pause
  \begin{align*}
  A=\left(\begin{array}{cc}
  1 & 0\\
  0 & 0
  \end{array}\right),
  B=\left(\begin{array}{cc}
  0 & 1\\
  0 & 0
  \end{array}\right)\\
  AB=\left(\begin{array}{cc}
  0 & 1\\
  0 & 0
  \end{array}\right)\neq\left(\begin{array}{cc}
  0 & 0\\
  0 & 0
  \end{array}\right)=BA
  \end{align*}
  \end{myexample}
\end{frame}

\begin{frame}
  \frametitle{半群、有限群}
  \begin{mydef}[半群和單半群]
  \label{def:group:semigroup_and_monoid}
  一個代數結構 $\Group$ 若滿足 \ref{def:group:g1} 和 \ref{def:group:g2},則 $\GroupSet$ 稱為\textbf{半群 (Semigroup)}。若 $\Group$ 滿足 \ref{def:group:g1}、\ref{def:group:g2}、\ref{def:group:g3},則 $\GroupSet$ 稱為\textbf{單半群 (Monoid)}。
  \end{mydef}
  \begin{mydef}<2->[有限群]
  \label{def:group:finite_group}
  一個群 $\Group$ 若 $|G|<\infty$,則 $\GroupSet$ 稱為\textbf{有限群 (Finite group)}。
  \end{mydef}
\end{frame}

\section{子群}

\begin{frame}
  \frametitle{定義}
  \begin{mydef}[子群]
  \label{def:group:subgroup}
  一個群 $\Group$ 上,若 $\SubGroup$ 是一個群且 $\SubGroupSet\subseteq{\GroupSet}$、$\SubGroupSet\neq{\emptyset}$,則 $\SubGroup$ 被稱為 $G$ 的\textbf{子群 (Subgroup)}。
  \end{mydef}
  \begin{mynote*}<2->
  $\GroupRelation[\GroupSet]$ 和 $\GroupRelation[\SubGroupSet]$ 要是相同的。
  \end{mynote*}
\end{frame}

\begin{frame}
  \frametitle{子群簡化 (1)}
  \begin{mylma}[實用版子群]
  \label{lma:group:two_rule_subgroup}
  一個群 $\Group$ 上有一個非空子集 $\SubGroupSet$,$\SubGroupSet$ 滿足以下條件
  \begin{enumerate}
  \item 對於任意 $a,b\in{\SubGroupSet}$,$ab\in{\SubGroupSet}$
  \item 對於任意 $a\in{\SubGroupSet}$,存在 $a^{-1}\in{\SubGroupSet}$
  \end{enumerate}
  若且唯若 $\SubGroup$ 是一個子群。
  \end{mylma}
  \begin{mynote*}<2->
  事實上這兩個規則就是驗證 \ref{def:group:g1} 和 \ref{def:group:g4}。
  \end{mynote*}
\end{frame}

\begin{frame}
  \frametitle{子群簡化 (2)}
  \begin{mylma}[精簡版子群]
  \label{lma:group:one_rule_subgroup}
  一個群 $\Group$ 上有一個非空子集 $\SubGroupSet$,對於任意 $a,b\in{\SubGroupSet}$,$ab^{-1}\in{\SubGroupSet}$ 若且唯若 $\SubGroup$ 是一個子群。
  \end{mylma}
  \begin{mypropo}<2->
  \label{pro:group:two_subgroup_intersection}
  一個群 $\Group$ 上若有兩個子群 $\SubGroupSet_1$、$\SubGroupSet_2$,則 ${\SubGroupSet_1}\cap{\SubGroupSet_2}$ 是 $\GroupSet$ 的子群。
  \end{mypropo}
\end{frame}

\begin{frame}
  \frametitle{有限群}
  \begin{mythm}[有限子群]
  一個有限群 $\Group$ 上有一個非空子集 $\SubGroupSet$,對於任意 $a,b\in{\SubGroupSet}$,$ab\in{\SubGroupSet}$ 若且唯若 $\SubGroup$ 是一個子群。
  \end{mythm}
\end{frame}

\end{CJK}
\end{document}