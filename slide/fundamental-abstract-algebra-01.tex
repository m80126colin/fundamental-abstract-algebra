\documentclass[utf8]{beamer}

\usepackage{CJKutf8}
\usepackage{amsmath,amsfonts,amssymb}
\usepackage{tkz-graph,tkz-berge}
\usepackage{multicol}
\usepackage{xkeyval,xargs}

\usetheme{Boadilla}
\usecolortheme{crane}

\setbeamertemplate{items}[circle]

\begin{document}
\begin{CJK}{UTF8}{bkai}

\newtheorem{mydef}{定義}[section]
\newtheorem*{mydef*}{定義}
\newtheorem{myrule}[mydef]{原理}
\newtheorem{mythm}[mydef]{定理}
%\newtheorem{mypropo}[mydef]{性質}
\newtheorem{mycorol}[mydef]{推論}
%\newtheorem{myexample}[mydef]{範例}
%\newtheorem*{mynote*}{註}
\numberwithin{equation}{section}
\renewenvironment{proof}{\textbf{證明}}{\qed}
\newenvironment{mysol}{\textbf{解答}}{\qed}
\newenvironment{mypropo}{\begin{exampleblock}{性質}}{\end{exampleblock}}
\newenvironment{myexample}{\begin{exampleblock}{範例}}{\end{exampleblock}}
\newenvironment{mynote*}{\begin{alertblock}{註}}{\end{alertblock}}

%% === 載入符號表 ===
%% === Basic Number Symbols ===
\providecommandx*{\PosInt}{\ensuremath{{\mathbb{Z}^{+}}}}
\providecommandx*{\NonNegInt}{\ensuremath{\mathbb{N}}}
\providecommandx*{\Int}{\ensuremath{\mathbb{Z}}}
\providecommandx*{\Ration}{\ensuremath{\mathbb{Q}}}
\providecommandx*{\Irration}{\ensuremath{\mathbb{Q}_{c}}}
\providecommandx*{\Real}{\ensuremath{\mathbb{R}}}
\providecommandx*{\Comp}{\ensuremath{\mathbb{C}}}
\providecommandx*{\Mat}[3][1=n,2=n]{\ensuremath{\mathbb{M}_{#1\times{#2}}{\left({#3}\right)}}}
\providecommandx*{\FuncSet}[3][1=\mathbb{F},2=\mathbb{R},3=\mathbb{R}]{\ensuremath{#1{\left({#2,#3}\right)}}}
%% ===  ===
\providecommandx*{\True}{\ensuremath{\top}}
\providecommandx*{\False}{\ensuremath{\bot}}
%% ===  ===
\providecommandx*{\Func}[3]{\ensuremath{#1:#2\rightarrow{#3}}}
\providecommandx*{\DefSet}[2]{\ensuremath{\left\{{#1|#2}\right\}}}
\providecommandx*{\NSet}[3][1=A,2=\times,3=n]{\ensuremath{{#1}_{1}#2{{#1}_{2}}#2\ldots{#2{#1}_{#3}}}}

%% ===  ===
\providecommandx*{\defaultSet}{\ensuremath{S}}
\providecommandx*{\defaultRelation}[2][1=,2=\mathcal{R}]{\ensuremath{{#2}_{#1}}}
\providecommandx*{\defaultFirstRelation}{\ensuremath{\defaultRelation[1]}}
\providecommandx*{\defaultSecondRelation}{\ensuremath{\defaultRelation[2]}}
\providecommandx*{\defaultTwoRelation}{\ensuremath{\defaultFirstRelation,\defaultSecondRelation}}
\providecommandx*{\defaultNRelation}[2][1=1,2=n]{\ensuremath{\defaultRelation[#1],\ldots{},\defaultRelation[#2]}}
%% === Algebra ===
\providecommandx*{\defaultAdd}[1][1=\defaultSet]{\ensuremath{\defaultRelation[#1][+]}}
\providecommandx*{\defaultTimes}[1][1=\defaultSet]{\ensuremath{\defaultRelation[#1][\cdot]}}
\providecommandx*{\defaultZero}[1][1=\defaultSet]{\ensuremath{{0}_{#1}}}
\providecommandx*{\defaultUnit}[1][1=\defaultSet]{\ensuremath{{1}_{#1}}}
\providecommandx*{\Algebra}[2][1=\defaultSet,2=\defaultRelation]{\ensuremath{\left({#1,#2}\right)}}
\providecommandx*{\AddAlgebra}[1][1=\defaultSet]{\ensuremath{\Algebra[#1][{\defaultAdd[#1]}]}}
\providecommandx*{\TimesAlgebra}[1][1=\defaultSet]{\ensuremath{\Algebra[#1][{\defaultTimes[#1]}]}}
\providecommandx*{\GeneralAlgebra}[2][1=\defaultSet,2=\defaultNRelation]{\ensuremath{\Algebra[#1][#2]}}
\providecommandx*{\Op}[3][1=\defaultRelation]{\ensuremath{#2#1#3}}
%% === Group ===
\providecommandx*{\GroupSet}[1][1=G]{\ensuremath{#1}}
\providecommandx*{\GroupRelation}[2][1=\GroupSet,2=\cdot]{\ensuremath{#2_{#1}}}
\providecommandx*{\Group}[2][1=\GroupSet,2=\GroupRelation]{\ensuremath{\Algebra[#1][{#2[#1]}]}}
\providecommandx*{\GOp}[3][1=\GroupRelation]{\ensuremath{\Op[#1]{#2}{#3}}}

\title{Fundamental Abstract Algebra\\基礎抽象代數}
\author{許胖}
\institute[PCSH]{板燒高中}

\begin{frame}
  \titlepage
\end{frame}
\begin{frame}
  \frametitle{大綱}
  \tableofcontents
\end{frame}

\section{簡介}

\begin{frame}
  \frametitle{二元運算}
  \begin{mydef}[二元運算]<1->
  \label{def:algebra:binary_operation}
  一個函數 $\Func{\defaultRelation}{A\times{B}}{C}$,對於所有 $a\in{A}$、$b\in{B}$,存在\textbf{唯一}的 $c\in{C}$,使得 $\defaultRelation{(a,b)}=c$,我們稱 $\defaultRelation$ 是一個從 $A\times{B}$ 到 $C$ 的\textbf{二元運算 (Binary Operation)},此時記為 $a\defaultRelation{b}=c$。
  \end{mydef}
  \begin{mynote*}<2->
  若 $A=B=C=\defaultSet$,我們稱 $\defaultRelation$ 是定義在 $\defaultSet$ 上的二元運算。
  \end{mynote*}
\end{frame}

\begin{frame}
  \frametitle{範例}
  \begin{myexample}<1->下列為二元運算:
  \begin{enumerate}
  \item<2-> 整數加法 $\Func{+}{\Int\times{\Int}}{\Int}$
  \item<3-> 實數乘法 $\Func{\cdot}{\Real^2}{\Real}$
  \item<4-> 實係數矩陣乘法 $\Func{\cdot}{\Mat[m][n]{\Real}\times\Mat[n][p]{\Real}}{\Mat[m][p]{\Real}}$
  \item<5-> 充要條件 $\Func{\Leftrightarrow}{\mathcal{L}\times{\mathcal{L}}}{\left\{{\True,\False}\right\}}$
  \item<6-> $\mathcal{O}{(n^2)}$ 的 LCS 演算法 $\Func{\textsc{LCS}}{{{\{0,1\}}^{m}}\times{{\{0,1\}}^{n}}}{{\{0,1\}}^{k}}$
  \end{enumerate}
  \end{myexample}
\end{frame}

\begin{frame}
  \frametitle{n 元運算與代數系統}
  \begin{mydef}[n元運算]<1->
  \label{def:algebra:n_ary_operation}
  一個函數 $\defaultRelation: \NSet\rightarrow{B}$,對於所有 ${\left({\NSet[a][,]}\right)}\in{\NSet}$,存在\textbf{唯一}的 $b\in{B}$,使得 $\defaultRelation{\left({\NSet[a][,]}\right)}=b$,我們稱 $\defaultRelation$ 是一個從 $\NSet$ 到 $B$ 的\textbf{ n 元運算 (n-ary Operation)}。
  \end{mydef}
  \begin{mydef}[代數結構與代數系統]<2->
  \label{def:algebra:algebraic_structure}
  一\textbf{代數結構 (Algebraic Structure)} $\GeneralAlgebra$ 滿足以下條件
  \begin{enumerate}
  \item<3-> 有一非空集合 $\defaultSet$
  \item<4-> $\defaultNRelation$ 為定義在 $\defaultSet$ 上的二元運算
  \item<5-> 一系列的公理 $\mathcal{A}$
  \end{enumerate}
  若 $\defaultNRelation$ 為定義在 $\defaultSet$ 上的 n 元運算,則稱 $\GeneralAlgebra$ 為\textbf{代數系統 (Algebraic System)}。
  \end{mydef}
\end{frame}

\begin{frame}
  \frametitle{範例}
  \begin{myexample}<1->
  下列為代數結構:
  \begin{enumerate}
  \item<2-> 有理數與加法、乘法 $\Algebra[\Ration][+,\cdot]$
  \item<3-> 複係數矩陣乘法 $\Algebra[\Mat{\Comp}][\cdot]$
  \item<4-> 正整數與最大公因數 $\Algebra[\PosInt][\gcd]$,其中最大公因數為二元運算 $\Func{\gcd}{\PosInt\times{\PosInt}}{\PosInt}$
  \item<5-> 函數合成 $\Algebra[\FuncSet][\circ]$
  \end{enumerate}
  \end{myexample}
\end{frame}

\section{二元運算的性質}
\subsection{基本性質}

\begin{frame}
  \frametitle{基本性質 (1)}
  \begin{mydef}[封閉律]<1->
  \label{def:algebra:closure}
  一個代數結構 $\Algebra$ 中,若對於所有 $a,b\in{\defaultSet}$,使得 $\Op{a}{b}\in{S}$,則稱二元運算 $\defaultRelation$ 對 $\defaultSet$ 滿足\textbf{封閉律 (Closure)}。
  \end{mydef}
  \begin{mydef}[結合律]<2->
  \label{def:algebra:associativity}
  一個封閉的代數結構 $\Algebra$ 中,若對於所有 $a,b,c\in{\defaultSet}$,使得 $\Op{(\Op{a}{b})}{c}=\Op{a}{(\Op{b}{c})}$,則稱二元運算 $\defaultRelation$ 對 $\defaultSet$ 具有\textbf{結合律 (Associativity, Associative property)}。
  \end{mydef}
  \begin{mydef}[交換律]<3->
  \label{def:algebra:commutativity}
  一個封閉的代數結構 $\Algebra$ 中,若對於所有 $a,b\in{\defaultSet}$,使得 $\Op{a}{b}=\Op{b}{a}$,則稱二元運算 $\defaultRelation$ 對 $\defaultSet$ 具有\textbf{交換律 (Commutativity, Commutative property)}。
  \end{mydef}
\end{frame}

\begin{frame}
  \frametitle{基本性質 (2)}
  \begin{mydef}[吸收律]<1->
  \label{def:algebra:absorption_law}
  一個封閉的代數結構 $\Algebra[\defaultSet][\defaultTwoRelation]$ 中,若對於所有 $a,b\in{\defaultSet}$,使得
  \begin{align*}
  \Op[\defaultFirstRelation]{a}{({\Op[\defaultSecondRelation]{a}{b})}}=a\\
  \Op[\defaultSecondRelation]{a}{({\Op[\defaultFirstRelation]{a}{b})}}=a
  \end{align*}
  ,則稱二元運算 $\defaultTwoRelation$ 在 $\defaultSet$ 上滿足\textbf{吸收律 (Absorption law)}。
  \end{mydef}
  \begin{mynote*}<2->
  吸收律是定義在\textbf{一對}二元運算上,因此不能單獨定義一個運算子具有吸收律。
  \end{mynote*}
\end{frame}

\begin{frame}
  \frametitle{基本性質 (3)}
  \begin{mydef}[分配律]<1->
  \label{def:algebra:distributivity}
  一個封閉的代數結構 $\Algebra[\defaultSet][\defaultTwoRelation]$ 中,若對於所有 $a,b,c\in{\defaultSet}$,使得
  \begin{align*}
  \Op[\defaultFirstRelation]{a}{({\Op[\defaultSecondRelation]{b}{c})}}=\Op[\defaultSecondRelation]{({\Op[\defaultFirstRelation]{a}{b}})}{({\Op[\defaultFirstRelation]{a}{c}})}\\
  \Op[\defaultFirstRelation]{({\Op[\defaultSecondRelation]{b}{c})}}{a}=\Op[\defaultSecondRelation]{({\Op[\defaultFirstRelation]{b}{a}})}{({\Op[\defaultFirstRelation]{c}{a}})}
  \end{align*}
  ,則稱二元運算 $\defaultFirstRelation$ 在 $\defaultSet$ 上對 $\defaultSecondRelation$ 具有\textbf{分配律 (Distributivity, Distributive property)}。
  \end{mydef}
  \begin{mynote*}<2->
  儘管 $\defaultFirstRelation$ 對 $\defaultSecondRelation$ 有分配律,但 $\defaultSecondRelation$ \textbf{未必}對 $\defaultFirstRelation$ 有分配律。
  \end{mynote*}
\end{frame}

\subsection{單位元素}

\begin{frame}
  \frametitle{單位元素 (1)}
  \begin{mydef}[單位元素]
  \label{def:algebra:identity}
  一個封閉的代數結構 $\Algebra$ 中,若
  \begin{itemize}
  \item 存在 $e_l\in{\defaultSet}$,對所有 $a\in{\defaultSet}$,$\Op{e_l}{a}=a$,則 $e_l$ 為\textbf{左單位元素 (Left identity)}
  \item 存在 $e_r\in{\defaultSet}$,對所有 $a\in{\defaultSet}$,$\Op{a}{e_r}=a$,則 $e_r$ 為\textbf{右單位元素 (Right identity)}
  \item 存在 $e\in{\defaultSet}$,對所有 $a\in{\defaultSet}$,$\Op{e}{a}=\Op{a}{e}=a$,則 $e$ 為\textbf{單位元素 (Identity)}
  \end{itemize}
  \end{mydef}
\end{frame}

\begin{frame}
  \frametitle{單位元素 (2)}
  \begin{mythm}[單位元素存在性]<1->
  \label{thm:algebra:identity_existence}
  一個封閉的代數結構 $\Algebra$ 中,若存在左單位元素 $e_l$、右單位元素 $e_r$,則 $e_l=e_r$,即單位元素存在。
  \end{mythm}
  \begin{mythm}[單位元素唯一性]<2->
  \label{thm:algebra:identity_uniqueness}
  一個封閉的代數結構 $\Algebra$ 中,若存在單位元素,則單位元素唯一。
  \end{mythm}
\end{frame}

\subsection{反元素}

\begin{frame}
  \frametitle{反元素 (1)}
  \begin{mydef}[反元素]
  \label{def:algebra:inverse}
  一個封閉的代數結構 $\Algebra$ 存在單位元素 $e\in{\defaultSet}$,若對 $a\in{\defaultSet}$,
  \begin{itemize}
  \item 存在 $b_l\in{\defaultSet}$,$\Op{b_l}{a}=e$,則 $b_l$ 稱為 $a$ 的\textbf{左反元素 (Left inverse)}
  \item 存在 $b_r\in{\defaultSet}$,$\Op{a}{b_r}=e$,則 $b_r$ 稱為 $a$ 的\textbf{右反元素 (Right inverse)}
  \item 存在 $b\in{\defaultSet}$,$\Op{b}{a}=\Op{a}{b}=e$,則 $b$ 稱為 $a$ 的\textbf{反元素 (Inverse)},$a$ 又稱\textbf{可逆元素 (Invertible element)}
  \end{itemize}
  若對所有 $a\in{\defaultSet}$ 都有反元素,則稱 $\defaultRelation$ 在 $\defaultSet$ 上有\textbf{反元素 (Inverse property)}。
  \end{mydef}
\end{frame}

\begin{frame}
  \frametitle{反元素 (2)}
  \begin{mypropo}<1->
  \label{pro:algebra:identity_has_inverse}
  一個封閉的代數結構 $\Algebra$ 存在單位元素 $e$,則 $e$ 的反元素為 $e$。
  \end{mypropo}
  \begin{mythm}[反元素存在性*]<2->
  \label{thm:algebra:inverse_existence}
  一個封閉的代數結構 $\Algebra$ 存在單位元素 $e$,且 $\defaultRelation$ 具有結合律,若 $a\in{\defaultSet}$ 存在左反元素 $b_l$,右反元素 $b_r$,則 $b_l=b_r$,即反元素存在。
  \end{mythm}
  \begin{mythm}[反元素唯一性*]<3->
  \label{thm:algebra:inverse_uniqueness}
  一個封閉的代數結構 $\Algebra$ 存在單位元素 $e$,且 $\defaultRelation$ 具有結合律,若 $a\in{\defaultSet}$ 存在反元素,則反元素唯一。
  \end{mythm}
\end{frame}

\begin{frame}
  \frametitle{反元素 (3)}
  \begin{mythm}
  \label{thm:algebra:left_identity_inverse}
  一個封閉的代數結構 $\Algebra$ 若滿足結合律,則以下兩個敘述是等價的:
  \begin{enumerate}
  \item<2-> \label{thm:algebra:left_id_inv_first}
      \begin{enumerate}
      \item 有左單位元素 $e_l$
      \item 對所有 $a\in{\defaultSet}$,存在左反元素
      \end{enumerate}
  \item<3-> \label{thm:algebra:left_id_inv_second}
      \begin{enumerate}
      \item $e_l$ 是單位元素
      \item (*)\label{exe:algebra:left_identity_inverse}對所有 $a\in{\defaultSet}$,存在反元素
      \end{enumerate}
  \end{enumerate}
  \end{mythm}
\end{frame}

\subsection{零元素與零因子}

\begin{frame}
  \frametitle{零元素 (1)}
  \begin{mydef}[零元素]<1->
  \label{def:algebra:zero_element}
  一個封閉的代數結構 $\Algebra$ 中,若
  \begin{itemize}
  \item 存在 $z_l\in{\defaultSet}$,對所有 $a\in{\defaultSet}$,$\Op{z_l}{a}=z_l$,則 $z_l$ 為\textbf{左零元素 (Left zero element)}
  \item 存在 $z_r\in{\defaultSet}$,對所有 $a\in{\defaultSet}$,$\Op{a}{z_r}=z_r$,則 $z_r$ 為\textbf{右零元素 (Right zero element)}
  \item 存在 $z\in{\defaultSet}$,對所有 $a\in{\defaultSet}$,$\Op{z}{a}=\Op{a}{z}=z$,則 $z$ 為\textbf{零元素 (Zero element)}
  \end{itemize}
  \end{mydef}
  \begin{mynote*}<2->
  零元素又稱\textbf{吸收元素 (Absorbing element)}。
  \end{mynote*}
\end{frame}

\begin{frame}
  \frametitle{零元素 (2)}
  \begin{mythm}[零元素存在性*]<1->
  \label{thm:algebra:zero_existence}
  一個封閉的代數結構 $\Algebra$ 具有結合律,若存在左零元素 $z_l$,右零元素 $z_r$,則 $z_l=z_r$,即零元素存在。
  \end{mythm}
  \begin{mythm}[零元素唯一性*]<2->
  \label{thm:algebra:zero_uniqueness}
  一個封閉的代數結構 $\Algebra$ 具有結合律,若存在零元素,則零元素唯一。
  \end{mythm}
\end{frame}

\begin{frame}
  \frametitle{零元素 (3)}
  \begin{mythm}<1->
  \label{thm:algebra:identity_zero_distinct}
  一個封閉的代數結構 $\Algebra$ 有單位元素 $e$、零元素 $z$,若 $|\defaultSet|\geq{2}$,則 $e\neq{z}$。
  \end{mythm}
  \begin{mypropo}<2->(*)
  \label{pro:algebra:zero_no_inverse}
  一個封閉的代數結構 $\Algebra$ 存在單位元素 $e$,若 $\defaultRelation$ 在 $\defaultSet$ 上有零元素 $z$ 且 $z\neq{e}$,則 $z$ 沒有反元素。
  \end{mypropo}
\end{frame}

\begin{frame}
  \frametitle{零因子}
  \begin{mydef}[零因子]<1->
  \label{def:algebra:zero_divisor}
  一個封閉的代數結構 $\Algebra$ 中存在零元素 $z$,若 $a,b\in{\defaultSet}$ 且 $a,b\neq{z}$,使得 $\Op{a}{b}=z$,則 $a,b$ 稱為\textbf{零因子 (Zero divisor)}。
  \end{mydef}
  \begin{mythm}[零因子性質]<2->
  \label{thm:algebra:zero_divisor_no_inverse}
  一個封閉的代數結構 $\Algebra$ 滿足以下條件:
  \begin{itemize}
  \item 有結合律
  \item 存在單位元素 $e$
  \item 存在零元素 $z$
  \end{itemize}
  若 $a,b\in{\defaultSet}$ 是零因子,則 $a,b$ 沒有反元素。
  \end{mythm}
\end{frame}

\begin{frame}
  \frametitle{消去律 (1)}
  \begin{mydef}[消去律]
  \label{def:algebra:cancellation_law}
  一個封閉的代數結構 $\Algebra$ 中,對所有 $a,b,c\in{\defaultSet}$,若
  \begin{itemize}
  \item $\Op{a}{b}=\Op{a}{c}$ 可得到 $b=c$,則 $\defaultRelation$ 在 $\defaultSet$ 上有\textbf{左消去律 (Left cancellation law)}。
  \item $\Op{b}{a}=\Op{c}{a}$ 可得到 $b=c$,則 $\defaultRelation$ 在 $\defaultSet$ 上有\textbf{右消去律 (Right cancellation law)}。
  \item $\defaultRelation$ 滿足左消去律和右消去律,則 $\defaultRelation$ 在 $\defaultSet$ 上有\textbf{消去律 (Cancellation law)}。
  \end{itemize}
  \end{mydef}
\end{frame}

\begin{frame}
  \frametitle{消去律 (2)}
  \begin{mythm}[消去律性質]
  \label{thm:algebra:cancellation_law}
  一個封閉的代數結構 $\Algebra$ 滿足以下條件:
  \begin{itemize}
  \item 有結合律
  \item 有單位元素 $e$
  \item 對所有 $a\in{\defaultSet}$ 都有反元素
  \end{itemize}
  則 $\defaultRelation$ 在 $\defaultSet$ 上有消去律。
  \end{mythm}
\end{frame}

\begin{frame}
  \frametitle{消去律 (3)}
  \begin{mythm}
  \label{thm:algebra:identity_inverse_equilibrium}
  一個封閉的代數結構 $\Algebra$ 若滿足結合律,則以下兩個敘述是等價的:
  \begin{enumerate}
  \item<2-> \label{thm:algebra:id_inv_eq_first}
      \begin{enumerate}
      \item $e$ 是單位元素
      \item 對所有 $a\in{\defaultSet}$,存在反元素
      \end{enumerate}
  \item<3-> \label{thm:algebra:id_inv_eq_second} 對於任意 $a,b\in{\defaultSet}$,$x,y$ 是未知數,方程式 $\Op{a}{x}=b$ 和 $\Op{y}{a}=b$ 存在唯一解。
  \end{enumerate}
  \end{mythm}
\end{frame}

\subsection{符號簡化}

\begin{frame}
  \frametitle{符號簡化 (1)}
  \begin{mydef}[單位元素記號]
  \label{def:algebra:identity_notation}
  一個封閉的代數結構 $\Algebra$ 存在單位元素 $e$,若
  \begin{itemize}
  \item<1-> $\defaultRelation$ 為加法 $\defaultAdd$,則 $e$ 為\textbf{加法單位元素 (Additive identity)},此時 $e$ 記為 $\defaultZero$。
  \item<2-> $\defaultRelation$ 為乘法 $\defaultTimes$,則 $e$ 為\textbf{乘法單位元素 (Multiplicative identity)},$e$ 記為 $\defaultUnit$。
  \end{itemize}
  \end{mydef}
  \begin{mynote*}<3->
  \begin{enumerate}
  \item<3-> $\defaultAdd$ 不是真的代表實數或複數的加法運算,而是代表他在 $\defaultSet$ 上有類似我們常見的加法性質,因此用這個符號容易聯想;$\defaultTimes$ 亦然。
  \item<4-> 使用 $0$ 和 $1$ 做為記號只是方便我們去聯想他的性質,事實上並不是實數的「$0$」和「$1$」,只是單純的\textbf{符號}。
  \end{enumerate}
  \end{mynote*}
\end{frame}

\begin{frame}
  \frametitle{符號簡化 (2)}
  \begin{mydef}[反元素記號]
  \label{def:algebra:inverse_notation}
  一個封閉的代數結構 $\Algebra$ 存在單位元素 $e$,且所有 $a\in{\defaultSet}$ 均有反元素 $b\in{\defaultSet}$,若
  \begin{itemize}
  \item<1-> $\defaultRelation$ 為加法 $\defaultAdd$,則 $b$ 為\textbf{加法反元素 (Additive inverse)},此時 $b$ 記為 $-a$。
  \item<2-> $\defaultRelation$ 為乘法 $\defaultTimes$,則 $b$ 為\textbf{乘法反元素 (Multiplicative inverse)},此時 $b$ 記為 $a^{-1}$。
  \end{itemize}
  \end{mydef}
  \begin{mynote*}<3->
  同樣地,$-a$ 和 $a^{-1}$ 只是單純的符號,不要和減法與倒數搞混。
  \end{mynote*}
\end{frame}

\subsection{其他性質}

\begin{frame}
	\frametitle{冪等律}
	\begin{mydef}[冪等元素與冪等律]
	\label{def:algebra:idempotent}
	一個封閉的代數結構 $\Algebra$ 中,若有 $a\in{\defaultSet}$,使得 $\Op{a}{a}=a$,則 $a$ 稱為\textbf{冪等元素 (Idempotent element)}。若所有 $a\in{\defaultSet}$ 都是冪等元素,則稱二元運算 $\defaultRelation$ 在 $\defaultSet$ 上滿足\textbf{冪等律 (Idempotent)}。
	\end{mydef}
\end{frame}

\section{同態與同構}



\end{CJK}
\end{document}